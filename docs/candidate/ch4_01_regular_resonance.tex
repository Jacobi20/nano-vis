\section{Режим основного резонанса}

\subsection{Типовые аварии}


\subsection{Цель и постановка эксперимента}

Основным условием основного резонанса является совпадение частот собственных колебаний судна и частоты пика спектра волнения: $\omega_{roll} \approx \omega_{max}$.

Целью эксперимента является исследование влияние направленности волнения на возникновение резонанса и определение диапазона частот опасных для судна.

Для проведения эксперимента используется модель корабля класса \frqt{буксир} со следующими характеристиками:
\begin{itemize}
	\item	$L = 20\ m$
	\item	$B = 7\ m$
	\item	$T = 2\ m$
	\item	$D = 120\ 000\ kg$
\end{itemize}

Эксперимент состоит из следующих этапов:
\begin{enumerate}
	\item	Определение частот собственных колебаний судна 
			$\omega_{roll}$, $\omega_{pitch}$, $\omega_{heave}$ путем кренования на тихой воде.
	\item	Выбирается набор параметров внешних условий вида $(m, \omega_{max})$, для которых
			проводятся запись качки судна, в течении 120 секунд.
	\item	Этап №2	проводится для всех значений $m$ и $\omega_{max}$
	
			$m \in \left\lbrace 2,8,32,128,512 \right\rbrace$
			
			$\omega \in [0.8..3.0],\ \Delta\omega=0.1$ 

\end{enumerate}

\subsection{Анализ результатов эксперимента}
Согласно результатам эксперимента были получены следующие частоты собственных колебаний (временные диаграммы качки показаны на рис.~\ref{exp_rr_rolling}):
\begin{enumerate}
	\item	$ \omega_{roll} = 1.75 s^{-1}$
	\item	$ \omega_{pitch} = 2.15 s^{-1}$
	\item	$ \omega_{heave} = 2.87 s^{-1}$
\end{enumerate}

\begin{figure}[ht]
	\begin{center}
	\includegraphics[width=100mm]{exp_rr/rolling}
	\includegraphics[width=100mm]{exp_rr/pitching}
	\includegraphics[width=100mm]{exp_rr/heaving}
	\end{center}
	\caption{Временные диаграмма боротовой, килевой и вертикальной качки}
	\label{exp_rr_rolling}
\end{figure}

Для всех параметров морского волнения определяется 10\%-ная обеспеченность (90\%-ная квантиль) модуля угла бортовой качки и делится на корень общей энергию волнения. График данного соотношения (исходные результаты и обработанные фильтром Гаусса с окном размером 7) в зависимости от частоты пика спектра волнения и параметра формы углового распределения показан на рис.~\ref{rr_res}.

\begin{figure}[ht]
	\begin{center}
	\includegraphics[width=100mm]{exp_rr/roll_resonance}
	\includegraphics[width=100mm]{exp_rr/roll_resonance_f}
	\end{center}
	\caption{График соотношения 10\%-ной обеспеченность модуля угла бортовой качки к энергии волнения.}
	\label{rr_res}
\end{figure}

Как видно из графика пик резонанса приходится на диапазон частот 1.75-1.9, что приближенно соответствует собственной частоте бортовой качки судна.