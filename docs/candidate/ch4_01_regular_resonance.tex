\section{Режим основного резонанса}

\subsection{Типовые аварии}


\subsection{Цель и постановка эксперимента}

\textbf{Условием} основного резонанса является совпадение частот собственных колебаний судна и частоты пика спектра волнения: $\omega_{roll} \approx \omega_{max}$.

\textbf{Целью} эксперимента является исследование влияние направленности волнения на возникновение резонанса и определение диапазона частот опасных для судна.

Для проведения эксперимента используется модель корабля класса \frqt{буксир} со следующими характеристиками:
\begin{itemize}
	\item	$L = 20\ \text{м}$
	\item	$B = 7\ \text{м}$
	\item	$T = 2\ \text{м}$
	\item	$D = 120\ 000\ \text{кг}$
\end{itemize}

Эксперимент состоит из следующих этапов:
\begin{enumerate}
	\item	Определение частот собственных колебаний судна 
			$\omega_{roll}$, $\omega_{pitch}$, $\omega_{heave}$ путем кренования на тихой воде.
	\item	Выбирается набор параметров внешних условий вида $(m, \omega_{max})$, для которых
			проводятся запись качки судна, в течении 120 секунд.
	\item	Этап №2	проводится для всех значений $m$ и $\omega_{max}$
	
			$m \in \left\lbrace 2,8,32,128,512 \right\rbrace$
			
			$\omega \in [0.8..3.0],\ \Delta\omega=0.1\ \text{рад/с}$ 

\end{enumerate}

\subsection{Анализ результатов эксперимента}

Согласно результатам эксперимента были получены следующие частоты собственных колебаний (временные диаграммы качки показаны на рис.~\ref{exp_rr_rolling}):
\begin{itemize}
	\item	$ \omega_{roll} = 1.75\ \text{рад/с}$
	\item	$ \omega_{pitch} = 2.15\ \text{рад/с}$
	\item	$ \omega_{heave} = 2.87\ \text{рад/с}$
\end{itemize}

\begin{figure}[ht]
	\begin{center}
	\includegraphics[width=120mm]{exp_rr/rolling}
	\includegraphics[width=120mm]{exp_rr/pitching}
	\includegraphics[width=120mm]{exp_rr/heaving}
	\end{center}
	\caption{Временные диаграммы бортовой, килевой и вертикальной качки}
	\label{exp_rr_rolling}
\end{figure}

Для всех параметров морского волнения определяется 10\%-ная обеспеченность (90\%-ная квантиль) модуля угла бортовой качки и делится на корень общей энергию волнения. График данного соотношения в зависимости от частоты пика спектра волнения и параметра формы углового распределения показан на рис.~\ref{rr:data} и рис.~\ref{rr:dataf} (аппроксимированные данные).

Для аппроксимации данных воспроизведения резонансных явлений следует выбрать функцию, которая удволетворяет следующим требованиям: 
\begin{itemize}
	\item	функция должна иметь пик заданной высоты для заданной резонансной точки;
	\item	функция должна иметь параметр формы, который задает ширину полосы пропускания и характеризующий
			энергетическии потери.  
\end{itemize} 

Подходящей функцией является функция плотности распределением Коши (или функцию отклика), которую можно выразить следующим образом:

\begin{equation}
	R(\omega) = \frac{A \Delta \omega^2} {  (\Omega-\omega)^2 + \Delta \omega^2 } 
		 = \frac{A \left( \frac{\Omega}{Q} \right) ^2} {  (\Omega-\omega)^2 + \left( \frac{\Omega}{Q} \right) ^2 } 
\label{cauchy}
\end{equation}

$\omega$ --- частота, $\Omega_0$ --- резонансная частота, $A$ --- пиковая амплитуда, $\Delta \omega$ --- полоса пропускания, $Q$ --- добротность, величина, показывающая соотношение запасенной энергии к потерям энергии на сопротивление.

Анализ резонансных явлений осуществляется путем аппроксимации экспериментальных данных функцией $R(\omega)$ методом наименьших квадратов. Обработка результатов и построение графиков осуществляется пакетом Scilab \citep{scilab}.


На основании аппроксимации можно сделать выводы:
\begin{enumerate}
	\item	Параметр формы углового распределения значительно влияет на установившуюся амплитуду колебаний судна.
	\item 	При увеличении значения формы углового распределения энергетические потери незначительно снижаются.
\end{enumerate}

\begin{sidewaysfigure}
	\includegraphics[width=230mm]{exp_rr/roll_resonance}
	\caption{Экспериментальные данные соотношения 10\%-ной обеспеченность модуля угла бортовой качки к энергии волнения.}
	\label{rr:data}
\end{sidewaysfigure}

\begin{sidewaysfigure}
	\includegraphics[width=230mm]{exp_rr/roll_resonance_f}
	\caption{Аппроксимация данных соотношения 10\%-ной обеспеченность модуля угла бортовой качки к энергии волнения и параметры функции отклика.}
	\label{rr:dataf}
\end{sidewaysfigure}

%\begin{figure}[ht]
%	\begin{center}
%	\includegraphics[width=100mm]{exp_rr/roll_resonance}
%	\includegraphics[width=100mm]{exp_rr/roll_resonance_f}
%	\end{center}
%	\caption{График соотношения 10\%-ной обеспеченность модуля угла бортовой качки к энергии волнения.}
%	\label{rr_res}
%\end{figure}

Как видно из графика пик резонанса приходится на диапазон частот 1.75-1.9, что приближенно соответствует собственной частоте бортовой качки судна.