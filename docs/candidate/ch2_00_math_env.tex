\section{Математическая модель нерегулярного волнения}

Используемая модель волнения аппроксимирует морскую поверхность суперпозицией конечного числа гармонических волн \citep{lopatuhin2004}. Характер волнения определяется двумерным энергетическим спектром $S(\omega, \theta)$, $\omega$ --- угловая частота волны, а $\theta$ --- угол между направлением бега волны и направлением ветра. В литературе \citep{lopatuhin2004} рассматриваются различные модели спектров волнения, выраженные в терминах функции энергетической плотности от угловой частоты и направления распространения. Примером такого спектра является спектр Пирсона-Московица:

\begin{equation}
	\label{envmath2:spectra}
	\begin{split}
	S(\omega, \theta) &= S(\omega) \; \cos^m (\theta) / C \\
	S(\omega) &= \dfrac{\alpha g^2}{\omega^5} \exp \left[ 
	  -\dfrac{5}{4} \left( \dfrac{\omega_{max}}{\omega} \right)^{4} 
	\right]  \\
	C &= \int_{-\pi/2}^{\pi/2} cos^m\theta d\theta
	\end{split}
\end{equation}

где $m$ --- четное число, характеризует направленность спектра, $\omega_{max}$ --- частоту пика спектра, а $\alpha = 0.0081$.

На практике непрерывный спектр волнения можно аппроксимировать конечным числом гармоник, спектральное распределение которых аппроксимирует используемый спектр. С целью быстрого построения карты высот и скоростей для сложения гармоник используется двумерное быстрое преобразование Фурье (БПФ).
 
При использовании БПФ следует представить энергетический спектр как функцию от волнового вектора гармоники $\bvec{k}$:
$$ \bvec{k} = k\bvec{n},\quad 
k = \frac{\omega^2}{g} ,\quad 
\bvec{n}=(cos \theta, sin \theta), \quad 
\omega(\bvec{k}) = \sqrt{\lVert \bvec{k} \rVert g} $$

Рассмотрим полную энергию волнения и произведем замену переменных интегрирования, перейдя от $(\omega, \theta)$ к волновому вектору $\bvec{k}$:

\begin{equation}
	\begin{split}
	E &= \int\limits_0^{2\pi}
		 \int\limits_0^\infty S(\omega, \theta) d\omega d\theta
	  = \int\limits_0^{2\pi}
		 \int\limits_0^\infty 
			 \frac {S(\sqrt{kg}, \theta)g} {2\sqrt{kg}} dk d\theta\\
	  &= \iint\limits_{\mathbb{R}^2}
			 \frac{S(\omega(\bvec{k}), \theta(\bvec{k})) g}
				  {2\omega(\bvec{k}) \lVert \bvec{k} \rVert} d\bvec{k}
	  = \iint\limits_{\mathbb{R}^2} \hat{S}(\bvec{k})d\bvec{k}
	\end{split}
\end{equation}

Теперь можно аппроксимировать непрерывный энергетический спектр конечной суммой $N^2$ гармоник:

\begin{equation}
\label{longe-higgins}
\begin{split}
E = \iint\limits_{\mathbb{R}^2} \hat{S}(\bvec{k})d\bvec{k}
    \approx \sum_{i,j} \hat{S}(\bvec{k}_{i,j}) {\Delta k}^2 
    = \sum_{i,j} E_{i,j}\\
\bvec{k}_{i,j} = (i\Delta k, j\Delta k), \quad i,j = -N/2+1 \; .. \; N/2 
\end{split}
\end{equation}

Важно правильно выбрать значения $\Delta k$ и $N$, чтобы полученные гармоники достаточно плотно покрывали наиболее энергетически плотные участки спектра.

Построив конечный дискретный энергетический спектр, перейдем к амплитудному спектру. Амплитуда гармоники $a_{i,j} = \sqrt{2E_{i,j}} = \sqrt{2S(\bvec{k}_{i,j})\Delta k}$. Высоту морской поверхности в точке $\bvec{p}$ в момент времени $t$ можно представить в виде суперпозиции простых гармонических волн:

\begin{equation}
\label{wave_height}
\begin{split}
h_w(t,\bvec{p}) = \sum_{i,j} a_{i,j}cos(\bvec{p}\cdot \bvec{k}_{i,j} - \omega(\bvec{k}_{i,j})t + \delta_{i,j}) \\
= Re\left( \sum_{\bvec{k}} \tilde{h}(\bvec{k}, t)
 \exp(i\bvec{k} \cdot \bvec{p}) \right)
\end{split}
\end{equation}
где значения $\delta_{i,j}$, задающие фазу каждой гармоники, выбираются случайно.

Для быстрого сложения $N^2$ гармоник было применено быстрое обратное двумерное преобразование Фурье. С его помощью эффективно вычисляются значения высот морской поверхности в узлах квадратной регулярной решетки размером $N \times N$ и пространственной протяженностью $\dfrac{2\pi}{\Delta k}$. В используемой симуляции $N = 512$.

На рис. \ref{waveplot} представлены планшеты ядра БПФ и карт высот морского волнения для различных значение параметра формы углового распределения.

Давление в точке $\bvec{p}$ на глубине $d$ с учетом волновой поправки выражается следующим образом:
\begin{equation}
\label{wave_height}
	p_w(t,\bvec{p},d) = \sum_{i,j}
		\gamma e^{-\abs{\bvec{k}_{i,j}}d} 
		a_{i,j}cos(\bvec{p}\cdot \bvec{k}_{i,j} 
		- \omega(\bvec{k}_{i,j})t 
		+ \delta_{i,j}) + d \gamma
\end{equation}

где $\gamma$ --- объемный вес воды. Согласно этой формуле чем больше длина волны, тем медленнее затухают колебания частиц с увеличением глубины.

Так как величина $\gamma e^{\abs{k}z}$ зависит от волнового числа, то вынести за знак суммы его невозможно что ставит под вопрос применимость БПФ для определения давления на каждой глубине.

Предположим, что:
\begin{equation}
\label{wave_height}
	p_w(t,\bvec{p},d) \approx
		\gamma e^{-k_{max}d} h_w(t,\bvec{p}) + d \gamma
\end{equation}

где $k_{max} = w_{max}^{2} / g$. Для определения погрешности такого допущения оценим значения давления на различных глубинах. Для этого найдем среднеквадратичное отклонение значения давления на разных глубинах
 ($d=0..16 \text{м}$) для разных частот пика максимума $w_{max}$. Результаты оценки ошибки продемонстрированы на рис.~\ref{wave_error}. Расчет показывает, что ошибка не превышает 10\%. 

В общем виде рассмотренная модель может строиться для частотно-направленных спектров произвольной формы (не обязательно \eqref{envmath2:spectra}); однако скорость ее сходимости для разных спектральных аппроксимаций может быть различной. При этом модель, основанная на БПФ, является, по сути, линейной. Однако она может учитывать нелинейные эффекты путем нелинейного безынерционного преобразования изначально сгенерированного гауссова случайного процесса (или поля) к требуемому одномерному закону распределения. Учитывая эффективность численной реализации процедуры БПФ, применение данной модели в составе виртуального полигона является более целесообразным, чем, например, моделей на основе авторегрессии-скользящего среднего \citep{dk8}. Это обусловлено их чувствительностью и нестабильностью для малых шагов сетки, что является важным для расчета гидродинамических сил и моментов интегрированием по корпусу. 



\begin{sidewaysfigure}
\begin{center}
	\includegraphics[width=250mm]{wave_plot/spec_all_rsz}
\end{center}
\caption{Планшеты ядра БПФ и карт высот морского волнения для различных значение параметра формы углового распределения}
\label{waveplot}
\end{sidewaysfigure}

\begin{sidewaysfigure}
	\includegraphics[width=230mm]{wave_plot_side}
	\caption{Оценка ошибки расчета давления приближенным способом: (a) --- профиль поверхностей равного давления при точном расчете; (б) --- профиль поверхностей равного давления при приближенном расчете; (в) --- стандартное отклонение ошибки расчета давления на разных глубинах для различных частот пика максимума; (г) --- относительная ошибка расчета давления на разных глубинах для различных частот пика максимума.}
	\label{wave_error}
\end{sidewaysfigure}

