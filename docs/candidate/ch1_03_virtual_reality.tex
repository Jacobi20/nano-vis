\section{Технологии визуализации и виртуальной реальности }

Научная визуализация как самостоятельная область исследований является относительно новым направлением в области информационных технологий и активно развивается примерно со второй половины 80-х годов. Прогресс в области научной визуализации, который достигнут в последние годы, был вызван активным развитием вычислительных и мультимедийных технологий, а также насущными потребностями науки и промышленности. Кроме того, текущая ситуация характеризуется стремительным ростом измеряемой и генерируемой информации, обработка которой без развитых средств визуализации практически невозможна. В наиболее общей постановке под научной визуализацией следует понимать методы и средства решения научных задач за счет привлечения к анализу данных способности человека видеть и интерпретировать изображения. В более строгой постановке научная визуализация – это междисциплинарное направление науки, основным назначением которого является визуализация многомерных динамических явлений и процессов.

Основной принцип научной визуализации предполагает, что человек гораздо лучше проникает в суть исследуемого явления, когда может «погрузиться» в пространство модели. Особенно эффект присутствия усиливается, когда человек получает возможность непосредственно манипулировать данными в этом пространстве. Такие технологии, получившие название виртуальной реальности [51–52], завоевывают все большую популярность в научном мире, однако их широкое распространение сдерживается дороговизной оборудования и определенными сложностями разработки прикладного программного обеспечения. Виртуальная реальность имитирует как воздействие, так и реакции на воздействие на моделируемую систему. Для создания убедительного комплекса ощущений реальности компьютерный синтез свойств и реакций виртуальной реальности производится в реальном времени. Системами виртуальной реальности называются устройства, которые более полно по сравнению с обычными компьютерными системами имитируют взаимодействие с виртуальной средой, путем воздействия на все пять (оптимально) имеющихся у человека органов чувств. Для этого могут использоваться различные технические решения, включая шлем виртуальной реальности [53], 3D мониторы, CAVE-системы [54] и пр.

Для отслеживания движений в системах виртуальной реальности используются различные технологии, которые преобразуют движения рук, головы или тела пользователя в координаты. Существуют следующие типы систем отслеживания движений [55]: маркерные оптические пассивные, маркерные оптические активные, магнитные, инерциальные, механические, безмаркерные, радиолокационные и ультразвуковые.

Исследования последних лет в области человеко-компьютерного взаимодействия привели к созданию нейрокомпьютерного интерфейса [56-58] – системы, разработанной для обмена информацией между мозгом и электронным устройством. В однонаправленных интерфейсах внешние устройства могут либо принимать сигналы от мозга, либо посылать ему сигналы (например, имитируя сетчатку глаза при восстановлении зрения электронным имплантатом). Двунаправленные интерфейсы позволяют мозгу и внешним устройствам обмениваться информацией в обоих направлениях. Однако, стоит отметить, что использование подобных интерфейсов в настоящее время порождает серьезные споры об этической оценке их использования [59–60].

Имитации тактильных и осязательных ощущений в системах виртуальной реальности применяются для решения задач виртуального прототипирования и эргономического проектирования, создания различных тренажеров, дистанционного управления роботами, в том числе микро- и нано-, системах создания виртуальных скульптур. Примерами таких устройств могут служить платформы подвижности (motion platform) [61] или перчатки виртуальной реальности [62]. Кроме того, для имитации различных эффектов в моделируемые интерфейсы пользователя могут встраиваться устройства так называемой силовой обратной связи, которые передают удары, вибрацию и т.д. Примером таких устройств могут служить рули или джойстики, реалистично передающие вибрацию движущегося автомобиля или летательного аппарата.

Альтернативной технологией для отображения научных данных являются программно-аппаратные комплексы класса TouchTable [63–64], которые представляют собой специализированные компьютеры с большим сенсорным монитором высокого разрешения. Управление работой комплекса производится прикосновениями пальцев к поверхности монитора, расположение которого в горизонтальной плоскости, т.е. в виде стола, делает удобным просмотр и анализ отображаемых пространственных данных группой пользователей.

Одной из наиболее важных особенностей научной визуализации является получение новых знаний; возможность его получения – один из главных критериев оценки эволюционного совершенства систем научной визуализации. Наиболее полно эти идеи нашли отражение в концепции когнитивной графики, как совокупности приемов и методов образного представления условий задачи, которые позволяют либо сразу увидеть решение, либо получить подсказку для его нахождения. Методы когнитивной графики используются в системах искусственного интеллекта, способных превращать текстовые описания задач в их образные представления, и при генерации текстовых описаний картин, возникающих во входных и выходных блоках интеллектуальных систем, а также в человеко-машинных системах, предназначенных для решения сложных, плохо формализуемых задач. Существует три основных задачи когнитивной компьютерной графики [65]:

\begin{itemize}
\item	создание таких моделей представления знаний, в которых была бы возможность однообразными средствами представлять как объекты, характерные для логического мышления, так и образы-картины, с которыми оперирует образное мышление;
\item	визуализация тех человеческих знаний, для которых пока невозможно подобрать текстовые описания;
\item	поиск путей перехода от наблюдаемых образов-картин к формулировке некоторой гипотезы о тех механизмах и процессах, которые скрыты за динамикой наблюдаемых картин.
\end{itemize}

Технологии когнитивной компьютерной графики основываются в целом на формальных методах искусственного интеллекта. В то же время в ряде случаев целесообразно использовать подходы к извлечению знаний на основе прямой эксплуатации способности человека видеть и интерпретировать изображения. Для этого в настоящее время активно продвигается метод интерактивной визуализации (computational steering) [66, 67]. Идея заключается в том, что пользователь по ходу процесса моделирования изменяет параметры системы, наблюдает и интерпретирует эффекты этого изменения. Технологически этот процесс реализуется за счет дополнительного слоя взаимодействия между пользователем и вычислительной средой, который преобразует поступающие от пользователя управляющие воздействия в набор параметров, которые в оперативном режиме воспринимаются вычислительной средой и позволяют производить вычисления в новых условиях без перезапуска вычислительного процесса.

Интерактивная визуализация с технической точки зрения является довольно сложной задачей, ключевыми проблемами которой для более широкого применения данной технологии являются: переориентация имеющегося ПО с пакетного режима работы на интерактивный режим, создание гибкой и расширяемой системы, поддержка «мягкого реального» времени [68]. При этом наибольшие проблемы с практическим воплощением методов интерактивной визуализации возникают в распределенных средах, как в силу пакетного (в основном) режима их работы, так и по причине наличия неконтролируемых (стохастических) накладных расходов при организации вычислений и передаче данных за счет коммунального характера среды. Вместе с этим, интерактивная визуализация в целом уменьшает нагрузку на распределенную среду; в таком режиме нет необходимости выполнять повторный запуск – все необходимые изменения можно сделать за одну сессию, что является несомненным преимуществом данного подхода.

Анализируя текущую ситуацию в области визуализации научных данных, можно резюмировать, что технологии научной визуализации стали неотъемлемым инструментом исследования для подавляющего большинства задач, связанных с компьютерным моделированием и экспериментом, и оказывают серьезное воздействие на сами методы научного познания. При этом широкое внедрение технологий научной визуализации связано с активным развитием сопутствующих технологий (качественное улучшение характеристик видеокарт, активное развитие пакетов прикладных программ и т.д.), однако в некоторых случаях сдерживается дороговизной и сложностью настройки предлагаемых решений (например, CAVE). В то же время проникновение логики e-Science в технологии научной визуализации привело к повышенной сложности использования соответствующего инструментария (в первую очередь, за счет неоднородности представляемых объектов и описывающих их данных); в целом для решения этой задачи необходимо привлечение интеллектуальных технологий. При этом, с одной стороны, необходимо учитывать качественную сторону проблемы – разнородность и распределенность данных, а с другой стороны, количественную – необходимость обработки значительных объемов данных. Первая проблема приводит к появлению специализированных методик визуализации распределенных научных данных (см., например, [69–70]). Вторая проблема связана с разработкой специализированных средств визуального анализа больших объемов данных (visual mining) (примерами таких работ могут служить исследования [71–72]). 

%С точки зрения организации UC, поддержка средств научной визуализации и виртуальной реальности является неотъемлемым элементом платформы VLUC. При этом главной проблемой является обеспечение ее функционирования в реальном времени в условиях жестких временных ограничений, диктуемых совокупно пользовательскими требованиями к принятию решений, ограничениями на производительность вычислительных процедур, генерирующих данные для визуализации, и характеристиками каналов передачи данных в систему визуализации.
