\section{Режим параметрического резонанса (судно расположено лагом к волне)}

\subsection{Типовые аварии}


\subsection{Цель и постановка эксперимента}

\textbf{Условием} возникновение параметрического резонанса является периодическое изменение характеристик остойчивости судна (\frqt{валкость}), что, при совпадении частот изменения остойчивости судна и частоты собственных бортовых колебаний может привести к возникновению резонансу.
Изменение характеристик остойчивости судна может происходить при изменении осадки (которое может быть обусловлено вертикальной качкой). Если выполняется условие
$2 \omega_{roll} \approx \omega_{heave} \approx \omega_{max}$, то судно может попасть резонанса, что может привести к опасному крену.

\textbf{Целью} эксперимента является определение и сравнение опасных диапазонов частот морского волнения.

Для проведения эксперимента используется модель корабля класса \frqt{катер} со следующими характеристиками:
\begin{itemize}
	\item	$L = 40\ \text{м}$
	\item	$B = 12\ \text{м}$
	\item	$T = 3\ \text{м}$
	\item	$D = 600\ 000\ \text{кг}$
\end{itemize}

Эксперимент состоит из следующих этапов:
\begin{enumerate}
	\item	Определение частот собственных колебаний судна 
			$\omega_{roll}$, $\omega_{pitch}$, $\omega_{heave}$ путем кренования на тихой воде.
	\item	Выбирается набор частот пика спектра волнения $(m, \omega_{max})$, для которых
			проводятся запись качки судна, в течении 900 секунд.
	\item	Этап №2	проводится для всех значений $\omega_{max}$
			
			$\omega \in [0.8..3.0],\ \Delta\omega=0.05$ 

\end{enumerate}

\subsection{Анализ результатов эксперимента}

Согласно результатам эксперимента были получены следующие частоты собственных колебаний (временные диаграммы качки показаны на рис.~\ref{exp_pr_rolling_lag}):
\begin{itemize}
	\item	$ \omega_{roll} = 1.14 \text{с}^{-1}$
	\item	$ \omega_{pitch} = 3.25 \text{с}^{-1}$
	\item	$ \omega_{heave} = 2.36 \text{с}^{-1}$
\end{itemize}

\begin{figure}[ht]
	\begin{center}
	\includegraphics[width=120mm]{exp_pr_lag/rolling}
	\includegraphics[width=120mm]{exp_pr_lag/pitching}
	\includegraphics[width=120mm]{exp_pr_lag/heaving}
	\end{center}
	\caption{Временные диаграммы бортовой, килевой и вертикальной качки.}
	\label{exp_pr_rolling_lag}
\end{figure}

Для всех параметров морского волнения определяется 10\%-ная обеспеченность (90\%-ная квантиль) модуля угла бортовой качки и делится на корень общей энергии волнения. График данного соотношения (исходные результаты) в зависимости от частоты пика спектра волнения показан на рис.~\ref{exp_pr_rolling_resonance_lag}.

\begin{figure}[ht]
	\begin{center}
	\includegraphics[width=120mm]{exp_pr_lag/roll_resonance}
	\end{center}
	\caption{Временные диаграммы бортовой, килевой и вертикальной качки.}
	\label{exp_pr_rolling_resonance_lag}
\end{figure}

На графике присутствуют два вида резонанса:
\begin{itemize}
	\item	Основной --- приходится на частоты $1.0..1.5$.
	\item	Параметрический --- приходится на частоты $2.1..2.2$.
\end{itemize}

