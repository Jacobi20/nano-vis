\section{Математические модели динамики внешней среды}

Основная задача математического моделирования динамики судна связана с обеспечением устойчивости его движения и формулируется как определение области в пространстве состояний судна, соответствующей требованиям эксплуатации, поиск границы этой области и связь параметров судна с критическими значениями параметров действующих возмущений \citep{dk1}. Исследование в такой постановке может быть выполнено только на основе анализа движения судна, находящегося под действием вызванных ветром и волнением сил. По этой причине для решения задачи построения модели динамики судна под воздействием внешней среды в экстремальных условиях эксплуатации необходимо разделить собственно математические (имитационные) модели изменчивости внешней среды и модели динамики морского объекта, находящегося в ней.

Непрерывный рост численности судов мирового флота, а также бурное освоение шельфовой зоны требуют не только умения предсказывать неблагоприятные погодные условия с той или иной заблаговременностью, но и определять количественные характеристики морских явлений редкой повторяемости, характеризующих экстремальную динамику внешней среды. К ним относятся скорость ветра, морское волнение, течения и уровень моря, а также их сочетания, возможные один раз в 100 или даже 1000 лет. Современная концепция получения информации о состоянии Мирового океана предполагает модельный подход на базе гидродинамического и статистического моделирования ветра, волнения, течений и уровня моря \citep{dk2}.

В инженерной практике экстремальные гидрометеорологические явления характеризуются расчетными сочетаниями скоростей ветра, параметров волнения, скоростей течений и уровня моря, возможными 1 раз в $T$ лет, где $T$ соответствует классу сооружения. Современная концепция получения информации об экстремальных гидрометеорологических явлениях основана на синтетическом подходе: на основе упорядоченных массивов метеорологической информации за несколько десятков лет выполняется гидродинамическое моделирование полей течений, морского волнения и уровня моря. Эти данные используются для идентификации стохастической модели, на основе которой выполняется экстраполяция расчетных характеристик на период повторяемости $T$. 

В рамках концепции \citep{dk3} основным источником данных об океанографических процессах (волнении, уровне, течениях) является сертифицированная (или, в метрологической практике – аттестованная) гидродинамическая модель динамики океана. По ней выполняются расчеты за непрерывный исторический период, обеспеченный данными наблюдений за атмосферными процессами (давлением, ветром и температурой воздуха). Гидродинамическая модель может быть интерпретирована как виртуальная измерительная система, верифицированная на основе разрозненных данных измерений, уже имеющихся в данном районе Мирового океана. Такой подход позволяет, используя данные реанализа метеорологических полей, получать информационные массивы океанографических характеристик непрерывной продолжительностью несколько десятков лет \citep{dk4}. Для статистического оценивания экстремальных характеристик, возникающих с определенной вероятностью, используется система стохастических моделей, описывающих совместную многомасштабную изменчивость пространственно-временных полей океанографических характеристик. Это дает возможность методом Монте-Карло воспроизвести ансамбль их реализаций, таким образом экстраполируя значения экстремумов на заданный временной интервал.

Следует отметить, что экстремальность гидрометеорологического явления по отношению к конкретному объекту определяется интегральной совокупностью всех факторов путем рассмотрения функций риска, специфичных для определенных классов морских объектов и сооружений. Это позволяет интерпретировать экстремальные гидрометеорологические явления не только в терминах скалярных характеристик (высоты волны, периода и пр.), а непосредственно в рамках формализма многомерных экстремумов.

Для имитационного моделирования конкретных воздействий на морские объекты и сооружения применяется иерархия моделей, позволяющая адекватно учесть многомасштабную изменчивость, обусловленную прохождением штормов. Так, для воспроизведения полей волнения в мелкомасштабном (секунды--часы) диапазоне изменчивости конкурентно (в зависимости от формы спектра) используются модели в форме полевой авторегрессии \citep{dk6} или в форме ортогональных разложений со случайными коэффициентами \citep{dk7}, в том числе с учетом нелинейности волнового профиля \citep{dk8}. Учет штормовой активности, в свою очередь, в диапазоне синоптической изменчивости осуществляется двумя альтернативными путями – посредством модели авторегрессии-скользящего среднего \citep{dk9} в синоптическом диапазоне (в терминах характерной высоты волны) или в терминах импульсной модели пространственно-временного поля в рамках лагранжева формализма \citep{dk10}. Учет климатической неоднородности, в свою очередь, требует применения другого класса моделей на основе ортогональных разложений по каноническому базису (естественные ортогональные функции) \citep{dk11}.

Для стохастического моделирования полей морского волнения может использоваться модель, основанная на использовании процессов авторегрессии-скользящего среднего \citep{p295I_31} в фиксированной точке пространства. Эта модель основывается на представлении процесса волнения как решения линейного дифференциального уравнения N-го порядка с постоянными коэффициентами и случайным входным сигналом.

Модель авто\-регрессии-скользящего среднего при высоком порядке по каждой из координат в рамках корреляционной теории является эквивалентной другой форме записи пространственно-временного случайного гауссова поля --- линейной модели Лонге-Хиггинса \citep{p295I_45}, котоое является формой разложения в дискретный ряд Фурье \citep{krogstad89}.

Для идентификации параметров моделей волнения необходимо знать корреляционную функцию или функцию спектральной плотности волновой поверхности. Спектральная плотность (или просто спектр) отражает свойства поля волнения при неизменных внешних факторах, как квазистационарного и квазиоднородного процесса. Изменения условий волнообразования связаны с прохождением барических образований (синоптическая изменчивость), годовой ритмикой (сезонная изменчивость) и долгопериодными вариациями циркуляционных процессов (межгодовая изменчивость), что позволяет определить волновой климат (или режим волнения) как ансамбль состояний волновой поверхности с учетом указанной изменчивости и описать его в терминах режимных статистических характеристик. Набор параметров из ансамбля промежутков квазистационарности характеризует волновой климат \citep{dk4}.

В целом волновой климат описывается ансамблем реализаций спектров за временной интервал, характеризуемый обычно несколькими десятками лет, в каждой точке  акватории. В силу такого подхода в качестве отдельных сценариев состояний моря выступают так называемые климатические спектры. Климатический спектр — это конкретный спектр, имеющий определенную вероятность (повторяемость появления в климатическом ансамбле) и отражающий ситуации, характерные для некоторого набора условий волнообразования \citep{dk4}. Для определения климатических спектров заданной повторяемости используются методы класса Монте-Карло \citep{dk8} или более сложные методы на основе многопараметрической оптимизации  \citep{dk5}.

Для отдельных задач моделирования внешних воздействий в режиме мелкомасштабной изменчивости используются физико-статистические методы. В частности, это связано с ситуациями, когда гидродинамические характеристики отдельной волны имеют определяющее влияние на объект, а априори восстановить ее профиль не представляется возможным. Такой подход характерен для моделирования волн-убийц и их воздействия на морские объекты и сооружения \citep{dk12}\citep{dk13}. Однако его применение далеко не всегда оправдано в силу ресурсоемкости вычислений.

Таким образом, рассмотренные модели позволяют воспроизводить изменчивость внешних возмущений, воздействующих на морской объект, в соответствии с их климатической повторяемостью для характерного района Мирового океана. Это дает возможность их использования в качестве входных данных для моделирования динамики морских объектов.