\chapter*{Введение}
\addcontentsline{toc}{chapter}{Введение}

Возможности изучения поведения сложных технических систем в экстремальных ситуациях экспериментальными
 методами существенно ограничены. Потому в настоящее время для этих целей активно применяется компьютерный
 эксперимент в реальном времени. Для интерпретации его результатов привлекаются технологии виртуальной
 реальности (ВР), обеспечивающие \frqt{погружение} исследователя в моделируемое явление с возможностью 
всестороннего наблюдения и анализа воспроизводимых закономерностей реального мира. В свою очередь, 
это стимулирует развитие нового класса комплексов проблемно-ориентированных программ для проведения
 вычислительного эксперимента –-- \textit{виртуальных полигонов} (ВП) для поддержки принятия решений в различных 
 областях науки и промышленности \footnote{ \citep{sloot111} }. Процесс проектирования и разработки ВП требует совокупного учета особенностей методов компьютерного моделирования в конкретной предметной области и соответствующих возможностей технологий ВР, включая специфику аппаратной реализации. Это достигается путем адаптации 
 методов математических моделей для формирования предметно-зависимых визуальных динамических сцен с 
 высоким уровнем реалистичности и достоверности. В отечественной науке существенный вклад в развитие 
 теоретических основ и практических решений в области технологий виртуальных полигонов внесен научными школами С.В. Клименко, Н.Н. Шаброва, М.В. Якобовского, Ю.М. Баяковского, М.В. Михайлюка, и ряда других исследователей.
 
 
Технологии ВП наиболее востребованы в направлениях, где проведение полномасштабных экспериментов экономически не выгодно или связано с существенными рисками. К таким областям, в частности, относится проектирование судов и технических средств освоения океана с повышенными требованиями к безопасности мореплавания, что требует изучения их поведения в разного рода аварийных ситуациях. Несмотря на то, что развитие аварийной ситуации является сложным многовариантным процессом, ретроспективный анализ известных инцидентов позволяет выделить условия, способствующие их возникновению, например, параметрические резонансы разной природы, потеря управляемости на гребне волны, захват судна волной (брочинг). В свою очередь, развитие каждой из вышеперечисленных ситуаций может усложняться за счет внутренних факторов (смещение навалочного груза, затопление отсеков, интенсивное обледенение и пр.). Как следствие, разнообразие и неоднозначность влияния экстремальных условий эксплуатации ограничивает возможности постановок экспериментов в опытовых бассейнах, и требует развития соответствующих проблемно-ориентированных программных комплексов на основе технологии ВП, что и определяет актуальность темы диссертации.

\ubf{Предметом исследования} является технология создания ВП применительно к задачам исследовательского проектирования морских объектов (МО) --- судов и средств освоения океана.

\ubf{Целью работы} является развитие методов формирования предметно-ориентированных динамических сцен на основе компьютерного моделирования экстремальной динамики МО под воздействием нерегулярных внешних возмущений, и разработка на их основе соответствующего математического и программного обеспечения ВП.

\ubf{Задачи исследования.} Достижение поставленной цели подразумевает решение следующих задач:
\begin{itemize}
\item Анализ существующих математических моделей поведения МО в экстремальных условиях эксплуатации, исходя из их применимости для формирования динамических сцен в ВП.
\item Разработка метода численного моделирования экстремальной динамики МО с шестью степенями свободы на нерегулярном трехмерном волнении с адаптацией к специфике использования в составе ВП; его алгоритмическая и программная реализация.
\item Разработка метода формирования динамических сцен с на основе численного моделирования динамики внешней среды и МО, и его адаптация для широкоэкранных систем ВР.  
\item Проектирование, разработка и отладка программного комплекса ВП; его развертывание на аппаратной инфраструктуре Центра ситуационного моделирования и визуализации (ЦСМВ СПбГУ ИТМО)\footnote{ЦСМВ – центр коллективного пользования СПбГУ ИТМО}.
\item Апробация ВП для проведения компьютерных экспериментов по исследованию экстремальной динамики МО в режиме основного и параметрического резонансов, а также в условиях брочинга\footnote{Неуправляемый разворот судна вследствие «захвата» волной, сопровождаемый сильным динамическим креном}. 
\end{itemize}

\ubf{Методы исследования} включают в себя методы вычислительной гидромеханики, теории вероятностей, математической статистики и имитационного моделирования, анализа алгоритмов и программ, обработки изображений и научной визуализации.


\ubf{Научную новизну} результатов работы определяют:
\begin{itemize}
	\item	Использование метода прямого моделирования динамики МО с шестью степенями свободы в нелинейной постановке, позволяющего унифицировать проведение компьютерного эксперимента для различных классов экстремальных явлений, с возможностью интерактивного управления через ВП.
	\item	Формирование реалистичных динамических сцен за счет применения метода численного интегрирования уравнений динамики МО на основе случайных сеток, обеспечивающих компенсацию ошибки вычислений и балансировку вычислительной нагрузки в условиях реального времени.
\end{itemize}


\ubf{Практическую ценность} работы составляют:
\begin{itemize}
\item	Комплект программной и эксплуатационной документации на программную систему для моделирования и визуализации динамики МО в экстремальных условиях эксплуатации\footnote{Свидетельство о регистрации программы для ЭВМ № 2011611381 – 2011}.
\item	Программно-аппаратный комплекс ВП ShipХ-DS, функционирующий на базе ЦСМВ СПбГУ ИТМО.
\end{itemize}


\ubf{На защиту выносятся:}

\begin{itemize}
\item	Метод формирования визуальных динамических сцен на основе численного моделирования нелинейной динамики МО c шестью степенями свободы на нерегулярном трехмерном волнении.
\item	Архитектура программного комплекса ВП для исследования МО в экстремальных условиях эксплуатации с поддержкой аппаратных возможностей широкоэкранных систем ВР.
\end{itemize}


\ubf{Достоверность научных результатов и выводов} обеспечивается строгостью наложенных ограничений предметной области, валидацией результатов моделирования путем сопоставления с классическими моделями корабельной гидродинамики, исследовательскими испытаниями работоспособности программно-аппаратного комплекса ВП на инфраструктуре ЦСМВ СПбГУ ИТМО, а также воспроизводимостью ряда нелинейных эффектов экстремальной динамики судна в ходе компьютерного эксперимента.

\ubf{Внедрение результатов работы.} Результаты работы нашли свое применение при выполнении проектов «Интеллектуальная система навигации и управления морским динамическим объектом в экстремальных условиях эксплуатации», «Интеллектуальные технологии поддержки процессов исследовательского проектирования судов и технических средств освоения океана», «Высокопроизводительный программный комплекс моделирования динамики корабля в экстремальных условиях эксплуатации», «Инструментальная технологическая среда для создания распределенных интеллектуальных систем управления сложными динамическими объектами» в рамках ФЦП «Научные и научно-педагогические кадры инновационной России на 2009 – 2013 годы», «Распределенные экстренные вычисления для поддержки принятия решений в критических ситуациях» в рамках реализации постановления Правительства РФ № 220 «О мерах по привлечению ведущих учёных в российские образовательные учреждения высшего профессионального образования», «Создание распределенной вычислительной среды на базе облачной архитектуры для построения и эксплуатации высокопроизводительных композитных приложений» рамках реализации постановления Правительства РФ №218 «О мерах государственной поддержки развития кооперации российских высших учебных заведений и организаций, реализующих комплексные проекты по созданию высокотехнологичного производства».  Результаты работ внедрены в производственную деятельность ЗАО «Фирма "АйТи". Информационные технологии».

\ubf{Апробация работы.} Изложенные в диссертации результаты обсуждались на семи международных и всероссийских научных конференциях, семинарах и совещаниях, включая Всероссийскую научно-техническую конференции "Интеллектуальные и информационные системы" (2009 г., Тула); IX и X ежегодные Международные конференции «Высокопроизводительные параллельные вычисления на кластерных системах» (2009 г., Владимир; 2010 г., Пермь); XVII Всероссийскую научно-методическая конференцию «Телематика 2010» (2010 г., Санкт-Петербург); V, VII Межвузовскую конференцию молодых ученых (2008 г., 2010 г., Санкт-Петербург); Всероссийскую конференцию «Технологии Microsoft в теории и практике программирования» (2010 г., Нижний Новгород), IV Международную конференцию по информатике MEDIAS (2011 г., Лимасол, Кипр).

\ubf{Публикации.} По теме диссертации опубликовано 10 печатных работ (из них 4 — в изданиях из перечня ведущих рецензируемых научных журналов и изданий, рекомендованных ВАК РФ).
