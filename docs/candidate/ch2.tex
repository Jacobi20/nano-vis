\chapter{Численное моделирование динамики судна на нерегулярном волнении}

%Следует отметить, что приведенные в данной главе модели динамики судна и 
%нерегулярного морского волнения являются лишь адаптацией существующих моделей
%с целью обеспечения требований к виртуальному полигону, которые были сформулированы в разделе \ref{ch1_1_5}.

Принципиальной задачей при организации численного моделирования в виртуальном полигоне является адаптация и доработка уже существующих методов, подходов и технологий к специфике постановок виртуальных экспериментов и отображения их результатов в реальном времени. Иными словами, встраивание моделей общего плана в виртуальный полигон приводит к определенному ограничению их индивидуальных возможностей за счет унификации способов представления входных и выходных параметров. Унификация необходима для того, чтобы можно было в единой форме проводить и описывать эксперименты для различных классов экстремальных явлений. В связи с этим, виртуальный полигон является не универсальной системой для получения произвольных характеристик объекта расчетным путем, а системой класса \frqt{испытательный стенд}, воспроизводящей только определенный диапазон условий. Как следствие, существующие модели требуют определенной методологической и технологической адаптации. Под методологической адаптацией понимается упрощение способов использования модели без существенных потерь качества воспроизведения экстремальных ситуаций (например, за счет различных параметризаций). Под технологической адаптацией --- применение вычислительных процедур и технологий, позволяющих выполнять расчеты в реальном времени. Вопросы адаптации численных моделей динамики судна и внешней среды к задачам виртуального полигона рассмотрены ниже.