\section{Технология виртуальных полигонов}

Безопасность морских объектов, таких как корабли (начиная от малых рыболовецких судов и заканчивая танкерами), платформы, прибрежные сооружения и т.д. по сей день является одной из приоритетных задач.
Встречающиеся экстремальные ситуации могут приводить к дискомфорту и ранениям экипажа, потере груза, повреждению или даже гибели судов.

В ряде случаев, использование реальных прототипов может быть опасно (испытание на реальном объекте), дорого, а также требовать серьезных затрат времени на подготовку и анализ результатов. Более того использование уменьшенных копий объектов в некоторых случаях не позволяет воспроизвести ряд явлений. [???????].

Таким образом, единственным способом моделирования рассматриваемых ситуаций является программное численное моделирование. В связи с этим вводится понятие виртуального полигона (Virtual test bed).
Виртуальный полигон – это программное средство, которое поддерживает виртуальное проектирование, анализ и прототипирование различных систем \citep{vtb_ship_ee} \citep{vtb_ILRO}. 

Виртуальный полигон должен удовлетворять следующим требованиями:
\begin{itemize}
	\item 	Реальное время выполнения (в идеале --- «сверхреальное»). «Сверхреальное» 
			время выполнения позволит перебирать несколько вариантов параллельно, что может быть полезно для предсказания поведения реальных объектов, а также производить массовую статистическую выборку.
	\item 	Интерактивность. В данном случае под интерактивностью подразумевается 
			возможность изменения параметров моделируемых объектов и получения 
			ответной реакции в реальном времени.
	\item 	Гибкость. Виртуальный полигон должен предоставлять набор конструктивных 
			и вспомогательных объектов для формирования задачи моделирования.  
			При этом объекты могут быть как конструктивными, так и вспомогательными. 
			Конструктивные отвечают за определение и формирование поведения и свойств 
			реального объекта, а вспомогательные позволяет формировать сценарии 
			модельных процессов и обработки результатов.
	% \item Возможность самонастройки и самоподстройки. 
	% Виртуальный полигон должен обеспечивать возможность гибкого 
	\item 	Возможность расширения: возможность добавления новых явлений, свойств и объектов.
\end{itemize}

Анализ задач моделирования динамики судна [???], а также виртуальных полигонов, использующихся в других научных и инженерных областях показал, что в состав виртуального полигона должны входить следующие компоненты:
\begin{itemize}
	\item   Подсистема человеко-машинного взаимодействия, в состав которой входят:
		\begin{itemize}
		\item 	Подсистема визуализации и отображения данных, 
				которая отвечает за отображение текстовой информации, 
				построения диаграмм а также двумерной, трехмерной и 
				стерео-визуализации моделируемого процесса.
		\item 	Подсистема ввода, которая может включать разнообразные 
				устройства ввода, начиная от простейшего текстового 
				интерфейса терминального типа и заканчивая интерфейсами 
				основанными на захвате движения [???].
		\end{itemize}
	\item	одсистема моделирования --- является центральной 
			частью виртуального полигона и отвечает за расчет пареметров 
			моделируемого объекта в каждый момент времени. Подсистема 
			моделирования может быть как монолитной, так и композитной. 
			В последнем случае, каждая компонента подсистемы моделирования 
			выполняет свою задачу и обменивается данными с другими компонентами.
	\item	Подсистема сценариев --- подсистема, которая отвечает за 
			формирование последовательностей, условий и дополнительных 
			действий, выполняемых в процессе моделирования.
\end{itemize}

Подсистемы виртуального полигона могут быть либо собраны в один програмный пакет, запускаемый на персональной ЭВМ так и сформированны в распределенный программно-аппаратный комплекс, который может состоять из:
\begin{itemize}
	\item 	Вычислительных серверов, кластеров, на которых 
			могут быть развернуты компоненты подсистемы моделирования.
	\item 	Графических станции, которые формируют изображение 
			целиком или по фрагментами. Разные графические могут 
			визуализировать моделируемый процесс с разных точек, 
			в разных масштабах и режимах отображения.
	\item 	Компонент ввода, которые могут включать в себя клавиатуры, 
			мыши, сенсорные панели, системы захвата движения, нейро-интерфейсы и т.п.
\end{itemize}





