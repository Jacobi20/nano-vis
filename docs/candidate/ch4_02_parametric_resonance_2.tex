\section{Режим параметрического резонанса на встречном волнении}

\subsection{Типовые аварии}


\subsection{Цель и постановка эксперимента}

Условием возникновение параметрического резонанса является периодическое изменение характеристик остойчивости судна (\frqt{валкость}), что, при совпадении частот изменения остойчивости судна и частоты собственных бортовых колебаний может привести к возникновению резонансу.
Изменение характеристик остойчивости судна может происходить в случае, если длина волны (гармоники морского волнения, соответствующей $\omega_{max}$) соизмерима с размерами судна $\lambda \approx L$. В этом случае судно периодически оказывается то на вершине волны (остойчивость уменьшается), то на подошве (остойчивость увеличивается). Для возникновения резонанса необходимо, чтобы кажущаяся частота волн $\omega_{encounter} \approx 2\omega_{roll}$.

Для проведения эксперимента используется модель корабля класса \frqt{катер} со следующими характеристиками:
\begin{itemize}
	\item	$L = 40\ m$
	\item	$B = 12\ m$
	\item	$T = 3\ m$
	\item	$D = 600\ 000\ kg$
\end{itemize}

Для проведения эксперимента фиксируются следующие условия морского волнения, которые соответствуют 5-ти баллам по шкале Бофорта:
\begin{itemize}
	\item	$\omega_{max} = 1.2$
	\item	$\gamma = 6.0$
	\item	$m = 64$
\end{itemize}

Для обеспечения заданной скорости корабля $V$ и курсового угла $\phi$ (воздействие морских волн может привести к развороту судна) используется упрощенная модель автоматизированной системы управления. Если скорости меньше заданной, то буксировочная сила увеличивается, если больше --- уменьшается. Если происходит отклонение от курса, то буксировочная сила прикладывается под углом к продольной оси суда, что приводит к возникновению крутящего момента и возврату судна на заданный курс.

Эксперимент состоит из следующих этапов:
\begin{enumerate}
	\item	Определение частот собственных колебаний судна 
			$\omega_{roll}$, $\omega_{pitch}$, $\omega_{heave}$ путем кренования на тихой воде.
	\item	Осуществляется численное моделирование качки судна для заданной скорости $V$, длительность моделирования равна 10 минутам.
	\item	Этап №2	проводится для всех значений $V$ в диапазоне $7..12\ ms^{-1}$, $\Delta V=0.125\ ms^-1$ 
\end{enumerate}

\subsection{Анализ результатов эксперимента}

Согласно результатам эксперимента были получены следующие частоты собственных колебаний (временные диаграммы качки показаны на рис.~\ref{exp_pr_rolling}):
\begin{itemize}
	\item	$ \omega_{roll} = 1.09 s^{-1}$
	\item	$ \omega_{pitch} = 1.83 s^{-1}$
	\item	$ \omega_{heave} = 2.22 s^{-1}$
\end{itemize}

\begin{figure}[ht]
	\begin{center}
	\includegraphics[width=120mm]{exp_pr_head_seas/rolling}
	\includegraphics[width=120mm]{exp_pr_head_seas/pitching}
	\includegraphics[width=120mm]{exp_pr_head_seas/heaving}
	\end{center}
	\caption{Временные диаграммы бортовой, килевой и вертикальной качки.}
	\label{exp_pr_rolling}
\end{figure}

Условие резонанса можно переписать как:
\begin{equation}
	\omega_{encounter} = 2 \omega_{roll} = \omega_{max} \left(  1 + \frac{V}{c_{max}}  \right)
\end{equation}
Так как $c_{max} = g / \omega_{max}$, то скорость, при которой возникает опасность резонанса $V_{pr}$ может быть выражена
\begin{equation}
	V_{pr} = \frac{g}{\omega_{max}} \frac{ 2 \omega_{roll} - \omega_{max} }{ \omega_{max}}
\end{equation}
