\section{Графическая подсистема}

\subsection{Общая структура графической подсистемы}

Графическая подсистема логически разделена на два уровня абстракции:
\begin{itemize}
	\item	Драйвер (driver).
	\item	Сцена (scene).
\end{itemize}

Драйвер --- представляет собой слой абстракции от конкретного графического API.
Таким образом, это позволяет внедрять новые графические API не затрагивая сцену.
Основными сущностями драйвера, которыми он оперирует и предоставляет ну уровень выше --- сцене, являются:
\begin{itemize}
	\item	Эффект (Effect) --- представляет собой совокупность 
			состояний графического конвейера и набора из вершинного и пиксельного шейдера.
	\item	Вершинный буфер (Vertex buffer) --- представляет собой совокупность из набора массива вершин и массива индексов.
	\item	Текстура (Texture) --- двумерное (2D texture), трехмерное (Volume texture) или кубическое (Cube texture) изображение.
	\item	Внеэкранная поверхность (Render Target) --- может быть использована 
			как область, куда может осуществляться визуализация, а затем как двумерная текстура.
\end{itemize}


Как было отмечено выше графическая подсистема позволяет отображать следующие виды объектов:

\begin{itemize}
	\item	Твердые объекты (Solids).
	\item	Водная поверхность (Water).
	\item	Объемные скалярные поля (Volume Scalar fields).
	\item	Отладочные линии (Debug lines).
	\item	Элементы пользовательского интерфейса и текст.
	\item	Источники света.
\end{itemize}

Для отображения твердых объектов используется технология Deferred Shading \citep{stalker}, \citep{killzone}.
Для для закраски твердых объектов используется модель освещения Кука-Торренса, которая является физически обоснованной \citep{cook_torrance} и дает наиболее реалистичные блики \citep{ngan}.
Детальное описание техники визуализации твердых объектов, построения теней и расчета освещения представлено в статье \citep{bezgodov08}.
Пример отображения твердых тел и построения теней представлен на рис.~\ref{solid_rendering}.

\begin{figure}[ht]
\begin{center}
%\includegraphics[width=150mm]{solid_rendering}
\LARGE{ДОБАВИТЬ КАРТИНКУ!}
\end{center}
\caption{Отображение твердых объектов и построение теней}
\label{solid_rendering}
\end{figure}



Для отображения элементов пользовательского интерфейса и текста используются списки наборы четырехугольников с наложенными текстурами.
Для отображения текста используются растровые шрифты, которые могут быть подготовлены из TrueType или OpenType шрифтов с использованием утилиты Bitmap Font Generator \citep{bmfont}.

Для отображения отладочной информации графическая система предоставляет функциональность по рендерингу трехмерных линий, 
а также следующих объектов, составляемых из линий:

\begin{itemize}
	\item	Стрелки (варьируются цвет, длина и форма наконечника)
	\item	Точки (изображаются как три взаимно-пересекающихся отрезка, варьируются размер и цвет)
	\item	Ориентированные по осям проволочный параллелепипед (варьируются размер и цвет)
\end{itemize}

Отладочные линии, а также их составные из линий объекты могут использованы для визуализации такой информации как:
\begin{itemize}
	\item	Силы и моменты действующие на судно.
	\item	Линии тока.
	\item	Траектории движения и графики.
\end{itemize}

На рис.~\ref{super_plot} представлен пример отладочной визуализации и текста.

\begin{figure}[ht]
\begin{center}
\includegraphics[width=110mm]{super_plot}
\end{center}
\caption{Отладочная визуализация и текст}
\label{super_plot}
\end{figure}

Для отображения трехмерных скалярных полей используется технология объемного рендеринга, которая 
заключается в трассировке луча в пространстве объемной текстуры и последующим накоплением 
значения трансфер-функции вычисленной для данной точки объема \citep{Engel01high-qualitypre-integrated}.
Пример объемного рендеринга на примере магнитных полей представлен на рис.~\ref{volume}

\begin{figure}[ht]
\begin{center}
\includegraphics[width=110mm]{volume}
\end{center}
\caption{Пример рендеринга скалярных полей}
\label{volume}
\end{figure}

%---------------------------------------------------------------------------------
%	WATER RENDERING :
%---------------------------------------------------------------------------------

\subsection{Технология отображения морского волнения}

