\section{Математические модели динамики морских объектов}

Основными целями изучения динамики корабля под воздействием внешних условий является обеспечение корабля качествами, в наилучшей степени обеспечивающими его использование по назначению. Центральное место среди них занимают мореходные качества корабля, под которыми понимают совокупность свойств, определяющих поведение корабля как плавающего сооружения в целом в различных условиях эксплуатации и при различных внешних воздействиях (в том числе экстремальных). Построение и анализ моделей динамики судна направлены в первую очередь на выявлении и изучение объективных закономерностей описывающих взаимодействием корабля с внешней средой, которые в одинаковой мере присущи мореходным качествам всех кораблей независимо от их индивидуальных различий. Знание этих закономерностей (особенно в форме хорошо апробированных методик и моделей) дает возможность предвидеть поведение корабля в различных условиях, а также указать те предупредительные меры, которые нужно предпринять, чтобы избежать гибельных для корабля последствий, что имеет большое значение, как для кораблестроителей, так и мореплавателей. С точки зрения постройки судов наличие моделей, обеспечивающих качественные оценки поведения судна под воздействием внешней среды (в том числе и экстремальных ситуаций) в зависимости от его размеров, формы корпуса, распределения грузов и т.п., предоставляет возможность обеспечить кораблю надлежащие мореходные качества еще  на этапе проектирования.

В зависимости от решаемых задач модели динамики судна могут описывать поведение судна на спокойной воде без анализа его движения при переходе из одного положения в другое (статическое рассмотрение различных положений корабля), а также непосредственно при движении. Такая классификация, как правило, определяет порядок проведения исследований. Несмотря на существенный прогресс в решении данных задач, достигнутый в последние десятилетия, заключающийся в широкой замене экспериментальных исследований в опытовых бассейнах компьютерным экспериментом, требуется систематическое изучение имеющихся моделей динамики судов под воздействием внешней среды (особенно применительно к новым типам судов), а также создание единой методологической основы для применения этих моделей.

В данном разделе рассматриваются существующие тенденции в области подходов, методов и моделей динамики корабля при взаимодействии с внешней средой, в том числе в сложных (экстремальных) условиях эксплуатации, а также моделирование внешних условий (ветра и волнения) эксплуатации морских динамических объектов. Основной акцент сделан на модели плавучести, остойчивости, непотопляемости и качки судов. Вопросы, касающиеся прочности корабля, тесно связанные с определением действующих на корабль сил, относятся к строительной механике корабля и в данном отчете не рассматриваются. Все качества судов, рассматриваемые в данном отчете, изучаются в предположении, что корабль обладает достаточной прочностью, что позволяет рассматривать его как абсолютно твердое тело.

\subsection{Классификация моделей динамики судов под воздействием внешних возмущений}
Основная задача математического моделирования динамики судна связана с обеспечением устойчивости его движения и формулируется как определение в пространстве состояний судна области, соответствующей требованиям эксплуатации, поиск границы этой области и связь параметров судна с критическими значениями параметров действующих возмущений \cite{dk1}. Исследование в такой постановке может быть выполнено только на основе анализа движения судна, находящегося под действием вызванных ветром и волнением сил. По этой причине для решения задачи построения модели динамики судна под воздействием внешней среды в экстремальных условиях эксплуатации необходимо разделить собственно математические (имитационные) модели изменчивости внешней среды, и модели динамики морского объекта, находящегося в ней.

\subsubsection{Математическое моделирование экстремальной динамики внешней среды}
Непрерывный рост численности судов мирового флота, а также бурное освоение шельфовой зоны требуют не только умения предсказывать неблагоприятные погодные условия с той или иной заблаговременностью, но и определять количественные характеристики морских явлений редкой повторяемости, характеризующих экстремальную динамику внешней среды. К ним относятся скорости ветра, морское волнение, течения и уровень моря, а также их сочетания, возможные один раз в 100 или даже 1000 лет. Современная концепция получения информации о состоянии Мирового океана предполагает модельный подход на базе гидродинамического и статистического моделирования ветра, волнения, течений и уровня моря [2].

В рамках концепции [3] основным источником данных об океанографических процессах (волнении, уровне, течениях) является сертифицированная (или, в метрологической практике – аттестованная) гидродинамическая модель динамики океана. По ней выполняются расчеты за непрерывный исторический период, обеспеченный данными наблюдений за атмосферными процессами (давлением, ветром и температурой воздуха). Гидродинамическая модель может быть интерпретирована как виртуальная измерительная система, верифицированная на основе разрозненных данных измерений, уже имеющихся в данном районе Мирового океана. Такой подход позволяет, используя данные реанализа метеорологических полей, получать информационные массивы океанографических характеристик непрерывной продолжительностью несколько десятков лет [4]. Для статистического оценивания экстремальных характеристик, возможных 1 раз в   лет, используется система стохастических моделей, описывающих совместную многомасштабную изменчивость пространственно-временных полей океанографических характеристик. Это дает возможность методом Монте-Карло воспроизвести ансамбль их реализаций, таким образом, экстраполируя значения экстремумов на заданный временной интервал.

Следует отметить, что экстремальность гидрометеорологического явления по отношению к конкретному объекту определяется интегральной совокупностью всех факторов путем рассмотрения функций риска, специфичных для определенных классов морских объектов и сооружений. Это позволяет интерпретировать экстремальные гидрометеорологические явления не только в терминах скалярных характеристик (высоты волны, периода и пр.), а непосредственно в рамках формализма многомерных экстремумов, который, описывает, например, климатические спектры морского волнения (как характерные состояния морской поверхности заданной обеспеченности) [4,5].
Для имитационного моделирования конкретных воздействий на морские объекты и сооружения применяется иерархия моделей, позволяющая адекватно учесть многомасштабную изменчивость, обусловленную прохождением штормов. Так, для воспроизведения полей волнения в мелкомасштабном (секунды-часы) диапазоне изменчивости конкурентно (в зависимости от формы спектра) используются модели в форме полевой авторегрессии [6] или в форме ортогональных разложений со случайными коэффициентами [7], в том числе, с учетом нелинейности волнового профиля [8]. Учет штормовой активности, в свою очередь, в диапазоне синоптической изменчивости осуществляется двумя альтернативными путями – посредством модели авторегрессии-скользящего среднего [9] в синоптическом диапазоне (в терминах характерной высоты волны), или в терминах импульсной модели пространственно-временного поля в рамках лагранжева формализма [10]. Учет климатической неоднородности, в свою очередь, требует применения другого класса моделей на основе ортогональных разложений по каноническому базису (естественные ортогональные функции) [11].

Для отдельных задач моделирования внешних воздействий в режиме мелкомасштабной изменчивости используются физико-статистические методы; в частности, это связано с ситуациями, когда гидродинамические характеристики отдельной волны имеют определяющее влияние на объект, а априори восстановить ее профиль не представляется возможным. В частности, такой подход характерен для моделирования волн-убийц и их воздействия на морские объекты и сооружения [12,13]. Однако его применение далеко не всегда оправдано в силу ресурсоемкости вычислений.

\subsubsection{Математическое моделирование динамики судна}
Качественное исследование динамики судна под воздействием внешней среды подразумевает построение рациональных математических моделей качки, позволяющих с одной стороны, достаточно точно отразить гидродинамические факторы, которые определяют действие волнующейся жидкости на морской объект, а с другой - выявить возможные в условиях эксплуатации режимы качки, представляющие реальную угрозу безопасности плавания. Построение таких моделей не является изолированной задачей, так как при их создании необходимо решение множества других задач, связанных с плавучестью, определением диаграммы статической и динамической остойчивости и т.п.  В настоящее время можно выделить четыре категории моделей динамики морских объектов:
\begin{itemize}
\item	Спектральные линейные и линеаризованные модели динамики судна.
\item	Нелинейные асимптотические модели динамики судна.
\item	Нелинейные численные модели динамики судна, основанные на уравнениях классической механики.
\item	Нелинейные численные модели динамики судна, основанные на уравнениях гидромеханики.
\end{itemize}

\subsection{Спектральные линейные и линеаризованные модели динамики судна}
Спектральные линейные и линеаризованные модели динамики судна начали активно развиваться в 60-е годы XX века в связи с накоплением знаний о спектральной структуре морского волнения. Данные модели основаны на допущении, что жидкость идеальна, ее волновое движение безвихревое (потенциальное), а амплитуды волн достаточно малы. Эти решения позволяют пренебречь квадратами вызванных скоростей и отыскать решение в рамках линейной теории. Чтобы использовать полученные результаты при определении сил и моментов, действующих на судно со стороны волнующейся жидкости, делается предположение, что и перемещения судна, вызванные волнением малой амплитуды, также малы.

Основными предпосылками линейной гидродинамической теории качки является относительная малость амплитуд набегающих волн и перемещений судна. Гидродинамическая теория учитывает не только воздействие волн на судно, но и определяет возмущения, вносимые судном в поле давлений волнующейся жидкости. В силу малости амплитуд набегающих волн волновое движение, обусловленное колебаниями судна, может быть представлено как при качке на тихой воде. Малость перемещений судна позволяет рассматривать дифрагированное волновое движение как дифракцию волн на неподвижном препятствии.

Предпосылки линейной гидродинамической теории качки непосредственно следуют из возможности представления действующих сил многомерными рядами Тейлора по динамическим координатам (перемещениям и скоростям) в окрестности положения равновесия судна на тихой воде. Затем удерживаются только первые члены, содержащие перемещения и скорости судна в степени не выше первой. Поскольку коэффициенты разложений определяются при значениях перемещений и скоростей судна в положении равновесия, то это и позволяет рассматривать силовое воздействие, обусловленное качкой, как таковое при колебаниях на тихой воде. Соответствующим образом определяется и силовое воздействие, обусловленное набегающими волнами.

Основные предпосылки линейной гидродинамической теории качки позволяют решать независимо друг от друга задачи об определении характеристик набегающих на судно волн, дифрагировании волнового движения, возмущенного движения жидкости, обусловленного вынужденными колебаниями судна на тихой воде. При этом используется принцип суперпозиции, согласно которому результирующее волновое движение определяется как сумма указанных волновых движений. 

В настоящее время эти модели лежат в основе ряда нормативных документов и методик, например [14,15]. В ряде случаев их можно обобщить и на случай линеаризованных моделей качки [1]. Однако в силу предположений об относительно малой амплитуде набегающих волн и перемещений судна, данные модели не применимы для исследования динамики судна в экстремальных ситуациях.


\subsection{Нелинейные асимптотические модели динамики судна}
В том случае, если динамика судна описывается классическими уравнениями механики, для ряда нелинейных систем в отдельных случаях можно строить их решения на основе асимптотических (аналитических) методов. Для получения аналитического решения в постановку задачи в форме уравнений механики вносятся определенные допущения: линеаризация основных уравнений и граничных условий по малому параметру – числу Фруда, приближенный учет вертикальных и продольных угловых колебаний судна и т.п. [16]. В работе [17] приведен пример аналитического решения системы дифференциальных уравнений качки одним из модификаций метода малого параметра. Однако такой подход имеет ограниченную применимость и может привести к потере новых решений, форма которых неизвестна заранее. Поэтому целесообразно использовать численные методы отыскания решений системы дифференциальных уравнений качки и оценки их устойчивости. Однако сложность таких решений может привести к большим затруднениям в правильной трактовке получаемых результатов: многообразие нелинейных факторов, которые учитываются общей математической моделью качки, в ряде случаев не позволяет выявить первопричину возникновения необычных колебаний или потерю их устойчивости. Чтобы осуществить это, допустимо, наряду с общей, воспользоваться некоторыми частными математическими моделями качки, учитывающими лишь часть нелинейных факторов и взаимосвязей между различными видами качки, влияние которых может быть изучено также аналитическими и экспериментальными методами. Несмотря на то, что для регулярного волнения этот подход позволяет в целом получать достаточно полную информацию о структуре нелинейных колебаний [17], прямой перенос его на область нерегулярных колебаний, например, представляя волновое возмущение в форме неканонического разложения, заставляет ограничиться рамками корреляционной теории [18], несмотря на очевидную негауссовость результирующего распределения. По этой причине в настоящее время подобные методы в основном носят качественный (прикидочный) характер и не используются для решения задач имитационного моделирования.

\subsection{Нелинейные численные модели динамики судна, основанные на уравнениях классической механики}
Многие важные задачи мореходности, интересные как в научном, так и в практическом отношении, не поддаются исследованию хорошо разработанными методами линейной теории. К задачам, полноценное решение которых требует применения нелинейных методов, т.е. учета конечности амплитуд качки, в первую очередь относятся:

\begin{itemize}
\item	Определение максимальной амплитуды бортовой качки в связи с обеспечением безопасности плавания.
\item	Расчет характеристик бортовой качки судов с очень малой метацентрической высотой.
\item	Исследование бортовой качки аварийного судна с отрицательной начальной остойчивостью.
\item	Расчет характеристик качки низкобортных судов.
\item	Оценка интенсивной заливаемости и оголения днища при качке на волнении.
\item	Расчет сопротивления воды движению судов на волнении.
\item	Исследование вопросов взаимного влияния отдельных видов продольной и поперечной качки судов.
\end{itemize}

Большинство перечисленных задач тесно связано с обеспечением безопасности судна в неблагоприятных условиях плавания, когда амплитуды качки близки к максимальным. Однако нелинейная теория охватывает не только вопросы интенсивной качки, но и любые другие задачи, требующие учета взаимного влияния различных гидромеханических факторов, которые в рамках линейной теории рассматриваются как независимые. В гидромеханической части задача нелинейной качки судов на регулярном и нерегулярном волнении решается следующим образом. В начале на базе приближенной гидромеханической теории качки конечной амплитуды составляют дифференциальные уравнения качки судна как твердого тела с шестью степенями свободы, движущегося с постоянной скоростью хода под произвольным курсовым углом к прогрессивным волнам относительно малой амплитуды, с такой степенью точности, определяемой малыми параметрами задачи, которая позволяет учесть все известные режимы нелинейной бортовой качки, включая субгармонические и хаотические. Это приводит системе связанных нелинейных неоднородных обыкновенных дифференциальных уравнений второго порядка, которые позволяют моделировать колебания морских судов на регулярном и нерегулярном волнении. Частные математические модели позволяют совместно исследовать поперечно-горизонтальные, вертикальные и бортовые колебания морских судов и других морских объектов, вводить вязкостные силы, экспериментировать с моделями нерегулярного морского волнения и т.д.

Использование численных методов позволило экстенсивным путем разрешить проблему исследования существенно нелинейных режимов качки, что привело к интенсивному развитию целого семейства моделей и соответствующих им программных реализаций: от иллюстрационных [19,20], на основе изолированных уравнений с постоянными коэффициентами, до достаточно детализированных, с переменными коэффициентами, которые рассчитываются непосредственно в процессе моделирования путем интегрирования по корпусу в рамках текущей ватерлинии в каждый момент времени [21,22].

Основные требования к моделям нелинейной качки судов заключатся в следующем: 
\begin{itemize}
	\item Даже при современном, весьма высоком уровне развития компьютерной техники нельзя чересчур усложнять общую задачу теории качки судов на волнении, иначе принципиальные трудности решения нелинейной граничной задачи по определению гидродинамических сил будут накладываться на трудности анализа кинематики движения судна, связанные с неустойчивыми режимами колебаний, хаотическими и другими, возможно, неучтенными эффектами.
	\item Для составления системы дифференциальных уравнений качки судна на волнении необходимо базироваться на такой гидродинамической модели процесса, которая без излишнего усложнения модели, позволила бы учесть все нелинейные факторы, которые могут хотя бы в принципе, повлиять на устойчивость колебаний судна, в особенности на потерю остойчивости, приводящую к опасным ситуациям в морских условиях.
\end{itemize}

Отдельный интерес представляют продольно-горизонтальные колебания, которые могут привести к явлению “захвата” судна волной с последующим поворотом его лагом к волнению (брочинг). Однако в рамках большинства моделей, которые целесообразно использовать для анализа нелинейных бортовых колебаний, явление брочинга описать невозможно. Для изучения таких явлений строятся специальные модели. Так, например, в работе [23,24] предложены независимые модели движения судна на попутном волнении и специальная модель бортовой качки, разработанная специально для изучения брочинга. 

Нелинейные модели качки и их модификации широко применяются в качестве базовых моделей при исследованиях динамики жидких грузов внутри судна [25,26], для исследования эффективности успокоителей качки и разработки алгоритмов их управления [27,28,29], для исследования эффекта параметрического резонанса [30].

Особых подходов требует исследование взаимодействия судна с аномальной волной [31]. Понятие “аномальные волны” получило в англоязычно литературе название “freak waves”,”rogue waves” или даже “killer waves”. Под аномальной волной понимают внезапно возникающую интенсивную волну, которая намного (в два и более раза) превосходит высоту фонового волнения. Помимо большого числа вопросов, которые возникают при оценке динамики судна при взаимодействии с аномальной волной с точки зрения моделирования, не менее острым остается вопрос физических основ возникновения таких волн и способов их математического описания [32].

\subsection{Нелинейные численные модели динамики судна, основанные на уравнениях гидромеханики}
Подход к моделированию динамики судна, основанный на уравнениях классической механики, не учитывает влияния присутствия и движения корабля на распределение гидродинамического давления в волнующейся жидкости, что не позволяет получить полное описание процесса качки и позволяет учесть лишь гидростатические силы и главную часть возмущающих сил. Гидродинамические инерционные и демпфирующие силы и дифракционные составляющие возмущающей силы остаются неучтенными; отчасти это корректируется введением соответствующих «виртуальных» членов (например, присоединенных масс и моментов). Для корректного учета этих составляющих «из первых принципов» необходимы модели динамики судна, основанные на уравнениях гидромеханики, которые позволяют решать задачи о движении качающегося корабля комплексно с учетом влияния корабля и жидкой среды как единой взаимодействующей системы. Данный подход является, по-видимому, наиболее адекватным с точки зрения воспроизведения динамики судна на морском волнении. Это положение связано с корректностью постановки задачи непосредственно в терминах уравнений гидромеханики, что позволяет не ограничиваться упрощенными параметризациями, характерными для уравнений классической механики. Однако такого рода модели не лишены недостатков, которые во многих случаях связаны с допущениями об идеальности жидкости, отсутствием вязкости и т.п., а также высокой вычислительной сложностью самих моделей. Гидромеханические модели можно эффективно применять для решения практически любых задач, связанных с движением судна, в том числе и специфических задач экстремальной динамики судна, моделирования качки поврежденного судна со свободными поверхностями в отсеках и т.п. [33,34].

Особые успехи применения гидродинамических моделей для решения задач динамики корабля можно отметить в последнее десятилетие, что связано во многом с развитием вычислительной техники, особенно технологий высокопроизводительных вычислений. Интересные результаты были получены в рамках проекта Министерства обороны США [35] по применению моделей, основанных на уравнении Рейнольдса (уравнения Навье-Стокса, осредненные по Рейнольдсу) к описанию движения судна под действием волн. В проекте использовались два пакета программ: UNCLE [36], разработанный Университете Миссисипи, и CFDSHIP-IOWA [37], разработанный в Университете Айова. Главная задача проекта состояла в предсказании поведения сложных геометрических объектов корабельной формы на воде. В результате были получены качественные модели, предсказывающие характеристики полей поверхностного давления и профиля свободной поверхности. Были также улучшены методы расчета свободной поверхности, вызванные поворотами судового руля. Интересные результаты по моделированию продольной и вертикальной качки корабля методами вычислительной гидродинамики были получены в работе [38].

Достаточно популярны пакетом гидродинамического моделирования для корабельных задач является LAMP (Large Amplitude Motion Program), который представляет собой пакет 3D-моделирования течений жидкости и решения проблемы взаимодействия твердого тела с волнами, в который также включены модели систем управления, заливаемости палубы, учет сил вязкости и другие эффекты. Пакет LAMP широко используется для решения задач нелинейной качки, параметрического резонанса и других. Детальное описание пакета LAMP и его приложений можно найти в [39,40]. 

В настоящее время такие гидродинамические модели для решения корабельных задач можно реализовать не только посредством специализированных программных систем, но и с помощью пакетов расчета динамики сплошных сред общего назначения, таких, например, как Fluent (здесь можно отметить и отечественную разработку Flow Vision [41]). Однако в последнем случае принципиальный аспект состоит в том, как корректно задать граничные условия на поверхности, учитывая физико-статистические свойства морского волнения [42].

Отдельный класс пакетов для гидродинамического моделирования задач корабельной гидромеханики представляют собой численные реализации моделей взаимодействия жидкости и твердого тела в условиях волнения (numerical wave tank [41-44]). Данные модели (и реализующие их пакеты) позволяют решать широкий класс задач, связанных с динамикой судов на волнении, в некоторых случаях позволяя получать результаты, сопоставимые по точности с экспериментами в опытовых бассейнах. Основной подход заключается в решении в расчетной области двух- или трехмерного уравнения Навье-Стокса с обновлением на каждом шаге граничных условий, вызванных наличием твердого тела и свободной поверхности.

Другим методом моделирования взаимодействия судна и жидкости может служить гидродинамика сглаженных частиц (Smoothed particle hydrodynamics) [45]. Метод сглаженных частиц является методом решения задач динамики жидкостей и твердых тел [46]. Метод сглаженных частиц позволяет решать задачи моделирования ударных волн [47], разрушающихся волн [48], расчета потока лавы [49], затопления [50] и т.д.

Метод сглаженных частиц является несеточным лагранжевым методом. Жидкость делится на частицы, каждая из которых является элементарным носителем свойств жидкости (масса, температура, заряд). Таким образом, значение любой физической величины в точке может быть определена как сумма значений всех частиц, умноженных на весовую функцию, которая зависит от расстояний от точки до частицы.
