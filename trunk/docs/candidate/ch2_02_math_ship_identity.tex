\section{Идентификация модели}

Cлучайное перестроение узлов может вносить в выражения \eqref{F=integral} и \eqref{M=integral} дополнительную вычислительную погрешность. Для оценки накапливаемой ошибки были проведены эксперименты по оценке чувствительности расчетных характеристик к количеству узлов сетки. На рис. \ref{drift} представлены фазовые портреты «численного дрейфа» (в плоскости $XY$) центра тяжести судна вокруг исходного положения (расчеты на тихой воде). На каждом графике показаны траектории движения центра тяжести судна для 10 запусков, каждый из которых длился 60 секунд.

Из рис. видно, что даже для достаточно грубых сеток погрешность не превышает 1 м, что является относительно небольшой величиной по сравнению с масштабами дрейфа под воздействием морского волнения.

\begin{figure}[ht]
\begin{center}
\includegraphics[width=140mm]{drift}
\end{center}
\caption{Фазовые портреты численного дрейфа судна в зависимости от количества точек $N$ }
\label{drift}
\end{figure}

Для идентификации модели рассмотрим процесс бортовой качки корабля на тихой воде.
В качестве испытательного образца возьмем судно класса "катер", который обладает следующими характеристиками:
\begin{description}
\item	Длина: $L = 40\ m$
\item	Ширина: $B = 7\ m$
\item	Вес: $D = 6 000 000\ N$
\end{description}

Для определения метацентрической высоты построим диаграмму статической остойчивости (ДСО) путем проведения виртуального кренования --- замера плеча восстанавливающего момента в зависимости от угла крена погруженного в воду судна. Результат кренования продемонстрирован на рис. \ref{stab}. Так как начальная метацентрическая высота равна дифференциалу ДСО в нулевой точке, то $h=1.03 м$

\begin{figure}[ht]
\begin{center}
\includegraphics[width=110mm]{stab}
\end{center}
\caption{Диаграмма статической остойчивости}
\label{stab}
\end{figure}

Определим период собственных бортовых колебаний согласно \citep{hanovich47} \eqref{roll_period}
\begin{equation}
	T_{roll} = \frac{0.80B}{\sqrt{h}} = 5.52\ s
	\label{roll_period}
\end{equation}

Момент инерции вокруг оси $X$ судна может быть приближенно вычислен по следующей формуле:
\begin{equation}
	I_{x} = (0.4B)^2 \frac{D}{g} \approx 4\ 700\ 000\ kg\cdot m^2
	\label{roll_torque}
\end{equation}



Момент сил сопротивления бортовым колебаниям может быть вычислен по формуле, полученной Бертеном \eqref{roll_torque}
\begin{equation}
	M_{roll} = kLB^4\dot{\alpha}^2
	\label{roll_torque}
\end{equation}

где $k$ лежит в диапазоне $10.0$ -- $20.0$. Для испытуемого судна возьмем значение $k$ равное $15.0$.

Составим дифференциальное уравнение бортовых колебаний судна:

\begin{equation}
	I_x \ddot{\alpha} + kLB^4\dot{\alpha}^2 + Dh \alpha = 0
	\label{roll_eq}
\end{equation}

Для идентификации модели корабль помещается на тихую воду с начальным креном $5$ градусов. Сравнение записи качки с результатом решения уравнение представлена на рис. \ref{expcmp}.

\begin{figure}[ht]
\begin{center}
\includegraphics[width=140mm]{expcmp}
\end{center}
\caption{Сравнение результатов численного эксперимента и решения уравнение бортовой качки}
\label{expcmp}
\end{figure}

Как видно и рисунка период и величина затухания бортовых колебаний соответствуют с высокой точностью, на основании чего можно сделать вывод, что численная модель движения судна позволяет воспроизводить эффекты частной модели.

