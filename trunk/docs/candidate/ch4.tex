\chapter{Применение ВП для воспроизведения экстремальных ситуаций}

\section{Общий подход анализа резонансных явлений}

Для рассмотрения резонансных свойств судна на нерегулярном волнении воспользуемся функцию распределением Коши (или функцию отклика) в виде:

\begin{equation}
	R(\omega) = \frac{A \Delta \omega^2} {  (\Omega-\omega)^2 + \Delta \omega^2 } 
		 = \frac{A \left( \frac{\Omega}{Q} \right) ^2} {  (\Omega-\omega)^2 + \left( \frac{\Omega}{Q} \right) ^2 } 
\label{cauchy}
\end{equation}

$\omega$ --- частота, $\Omega_0$ --- резонансная частота, $A$ --- пиковая амплитуда, $\Delta \omega$ --- полоса пропускания, $Q$ --- добротность, величина, показывающая соотношение запасенной энергии к потерям энергии на сопротивление.

Примеры резонансных кривых представлены на рис.~\ref{rc}.

\begin{figure}[h!]
\begin{center}
\includegraphics[width=90mm]{resonance_curve}
\end{center}
\caption{Примеры резонансных кривых для разных параметров выражения \eqref{cauchy}}.
\end{figure}

Анализ резонансных явлений осуществляется путем аппроксимации экспериментальных данных функцией $R(\omega)$ методом наименьших квадратов. Обработка результатов и построение графиков осуществляется пакетом Scilab \citep{scilab}.

