\chapter*{Заключение}
\addcontentsline{toc}{chapter}{Заключение}


В ходе выполнения диссертационного исследования автором лично был выполнен аналитический обзор в проблемной области диссертационной работы, проведена адаптация метода моделирования динамики судна на нерегулярном волнении к задаче формирования динамических сцен в ВП, проектирование, разработка и развертывание ВП, а также проведена серия экспериментов по воспроизведению таких экстремальных явлений, как основной и параметрический резонанс и брочинг. В диссертацию включены результаты, соответствующие личному участию автора, а именно:

\begin{itemize}

\item	развит метод численного моделирования экстремальной динамики МО с шестью степенями свободы на нерегулярном трехмерном волнении, основанный на интегрировании гидродинамических сил и моментов в нелинейной постановке на случайных сетках, допускающий интерактивное управление процессом вычислений на ВП;
\item	разработан метод формирования динамических сцен на основе численного моделирования динамики внешней среды и МО с учетом графических эффектов визуализации взволнованной поверхности моря и ее взаимодействия с корпусом объекта, адаптированный для применения в широкоэкранных системах ВР;
\item	разработана и детализирована архитектура ВП для изучения динамики МО в экстремальных условиях эксплуатации на основе модульного подхода к построению систем интерактивной визуализации;
\item		спроектирован и разработан программный комплекс ВП ShipX-DS, развернутый на инфраструктуре ЦСМВ СПбГУ ИТМО и продемонстрировавший свою работоспособность в ходе компьютерных экспериментов по исследованию экстремальной динамики МО в режиме основного и параметрического резонансов, а также в условиях брочинга.


%\item	развит метод численного моделирования экстремальной динамики МО с шестью степенями свободы на нерегулярном трехмерном волнении, основанный на интегрировании гидродинамических сил и моментов в нелинейной постановке на случайных сетках, допускающий интерактивное управление процессом вычислений в ВП;
%\item	разработан метод формирования динамических сцен на основе численного моделирования динамики внешней среды и МО, с учетом графических эффектов визуализации взволнованной поверхности моря и ее взаимодействия с корпусом объекта, адаптированный для применения в широкоэкранных системах ВР;
%\item	разработана и детализирована архитектура ВП для изучения динамики МО в экстремальных условиях эксплуатации на основе модульного подхода к построению систем интерактивной визуализации;
%\item	спроектирован и разработан программный комплекс ВП ShipX-DS, инсталлированный на инфраструктуре ЦСМВ СПбГУ ИТМО, и продемонстрировавший свою работоспособность в ходе компьютерных экспериментов по исследованию экстремальной динамики МО в режиме основного и параметрического резонансов, а также в условиях брочинга.

\end{itemize}