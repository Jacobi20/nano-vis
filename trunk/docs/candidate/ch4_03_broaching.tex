\section{Брочинг}

\subsection{Типовые аварии}

\subsection{Цель и постановка эксперимента}

Условием возникновения брочинга является:
\begin{itemize}
	\item	Движение судна в направлении распространения волн: $V=c$.
	\item	Длина волны ((соответствующая $\omega_{max}$)) находится в диапазоне от длины судна до удвоенной длины судна:
			: $L \leqslant c \leqslant 2L$.
	\item	Достаточно высокая степень регулярности волнения (зыбь): $\gamma > 10$.
\end{itemize}

Целью эксперимента является сбор статистики возникновения явления брочинга в одинаковых условиях (различие присутствует только в стохастических параметрах, таких как фазы гармоник морского волнения).

Для проведения используются следующие параметры волнения:
\begin{itemize}
	\item	$\omega_{max} = 1.2$
	\item	$\gamma = 20$
	\item	$m = 64$
\end{itemize}

Для проведения эксперимента используется модель корабля класса \frqt{буксир} со следующими характеристиками:
\begin{itemize}
	\item	$L = 20\ m$
	\item	$B = 7\ m$
	\item	$T = 2\ m$
	\item	$D = 120\ 000\ kg$
\end{itemize}

Эксперимент состоит из следующих этапов:
\begin{enumerate}
	\item	Определение буксировочной диаграммы судна --- зависимости скорости судна от приложенной силы. 
			На основе данной диаграммы выбирается постоянная буксировочная сила.
	\item	Проводится $N=200$ запусков (длительность $60 s$\footnote{60 секунд --- достаточное время развития явления брочинга для столь малого судна}) в одинаковых начальных условиях (различается лишь seed в генераторе случайных чисел \footnote{Используется вихрь Мерсенна \citep{mersenn_twister}})
\end{enumerate}


%----------------------------------------------------------------------

\subsection{Анализ результатов эксперимента}

\begin{figure}[ht]
	\begin{center}
	\LARGE{ДИАГРАММА БУКСИРОВОЧНОЙ СИЛЫ!}
	\end{center}
	\caption{Диаграмма буксировочной силы.}
	\label{tug_force}
\end{figure}

Частоте волнения соответствует фазовая скорость $c$

В ходе эксперимента были записаны все текущие параметры судна. На основе визуального анализа траектории были выделены следующие категории явления:

\begin{enumerate}
	\item	Присутствует значительное изменение курса с последующим восстановлением направление движения. 
			Явление может повториться вновь.
	\item	Изменение курса судна и отклонение и смещение судна от заданной траектории незначительно.
	\item	Крайне значительное изменение курса судна. Судно остается лагом к волне.
	\item	Значительное изменение курса судна.
	\item	Судно сохраняет общее направление движения, но периодически незначительно меняет курс.
\end{enumerate}

Результаты эксперимента сведены в таблицу.~\ref{table_broaching}.

\begin{table}[h!]
\caption{Сводная таблица результатов проведения эксперимента}
\label{table_broaching}
\begin{center}

	%\rotatebox{90}{ %это обеспечивает поворот любого объекта
	\begin{tabular}{|c|ccccc|}
	\hline
	\rotatebox{90}{Класс траектории}	&	
	\rotatebox{90}{Отклонение от курса ($\phi_{max}$)}		&	
	\rotatebox{90}{Угол крена ($\theta_{max}$)}	&	
	\rotatebox{90}{Скорость ($V$)}	&	
	\rotatebox{90}{Угловая скорость ($\theta'_{max}$)}	&	
	\%	\\
	\hline
	I		&	70.63	&	25.92	&	7.84	&	0.18	&	23	\\
	II		&	68.00	&	25.87	&	7.95	&	0.13	&	10	\\
	III		&	101.73	&	42.38	&	7.96	&	0.24	&	22	\\	
	IV		&	77.91	&	39.18	&	8.02	&	0.21	&	41	\\
	V		&	58.54	&	19.23	&	8.74	&	0.13	&	4	\\
	\hline
	\end{tabular}
	%}
	
\end{center}
\end{table}

Потеря управляемости возникает в 90\% случаев.
Наиболее опасные варианты развития брочинга (максимальный крен $\theta_{max}$ достигает 40 градусов) 
соответствуют классам траекторий III и IV. Суммарная вероятность наиболее опасного варианта развития события составляет 60\%.
