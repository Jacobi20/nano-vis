\section{Математическая модель нерегулярного волнения}

Используемая модель волнения аппроксимирует морскую поверхность суперпозицией конечного числа гармонических волн \cite{lopatuhin2004}. Характер волнения определяется двухмерным энергетическим спектром $S(\omega, \theta)$, $\omega$ -- угловая частота волны, а $\theta$ -- угол, образуемый направлением бега волны с направлением ветра. В литературе \cite{lopatuhin2004} рассматриваются различные модели спектров волнения, выраженные в терминах функции энергетической плотности от угловой частоты и направления распространения. Примером такого спектра является спектр Пирсона-Московица:

\begin{equation}
\begin{split}
S(\omega, \theta) &= S(\omega) \; C \cos^m\theta \\
S(\omega) &= \dfrac{\alpha g^2}{\omega^5} \exp \left[ 
  -\dfrac{5}{4} \left( \dfrac{\omega_{peak}}{\omega} \right)^{4} 
\right]  \\
C &= \int_{-\pi/2}^{\pi/2} cos^m\theta d\theta
\end{split}
\end{equation}, 
где $m$ характеризует направленность спектра, $\omega_{peak}$ -- частоту пика спектра, а $\alpha = 0.0081$.

На практике, непрерывный спектр волнения можно аппроксимировать конечным числом гармоник, спектральное распределение которых аппроксимирует используемый спектр. Для быстрого построения карты высот и скоростей для сложения гармоник используется двухмерное быстрое преобразование Фурье (БПФ) [].
 
При использовании БПФ оказалось удобнее представить энергетический спектр как функцию от волнового вектора гармоники $\bvec{k}$:
$$ \bvec{k} = k\bvec{n},\quad 
k = \frac{\omega^2}{g} ,\quad 
\bvec{n}=(cos \theta, sin \theta), \quad 
\omega(\bvec{k}) = \sqrt{\lVert \bvec{k} \rVert g} $$

Рассмотрим полную энергию волнения и произведем замену переменных интегрирования, перейдя от $(\omega, \theta)$ к волновому вектору $\bvec{k}$:
\begin{equation}
\begin{split}
E &= \int\limits_0^{2\pi}
     \int\limits_0^\infty S(\omega, \theta) d\omega d\theta
  = \int\limits_0^{2\pi}
     \int\limits_0^\infty 
         \frac {S(\sqrt{kg}, \theta)g} {2\sqrt{kg}} dk d\theta\\
  &= \iint\limits_{\mathbb{R}^2}
         \frac{S(\omega(\bvec{k}), \theta(\bvec{k})) g}
              {2\omega(\bvec{k}) \lVert \bvec{k} \rVert} d\bvec{k}
  = \iint\limits_{\mathbb{R}^2} \hat{S}(\bvec{k})d\bvec{k}
\end{split}
\end{equation}

Теперь можно аппроксимировать непрерывный энергетический спектр конечной суммой $N^2$ гармоник:
\begin{equation}
\begin{split}
E = \iint\limits_{\mathbb{R}^2} \hat{S}(\bvec{k})d\bvec{k}
    \approx \sum_{i,j} \hat{S}(\bvec{k}_{i,j}) {\Delta k}^2 
    = \sum_{i,j} E_{i,j}\\
\bvec{k}_{i,j} = (i\Delta k, j\Delta k), \quad i,j = -N/2+1 \; .. \; N/2 
\end{split}
\end{equation}

Важно правильно выбрать значения $\Delta k$ и $N$, чтобы полученные гармоники достаточно плотно покрывали наиболее энергетически плотные участки спектра.

Построив конечный дискретный энергетический спектр, перейдем к амплитудному спектру. Амплитуда гармоники $a_{i,j} = \sqrt{2E_{i,j}} = \sqrt{2S(\bvec{k}_{i,j})}\Delta k$. Высоту морской поверхности в точке $\bvec{p}$ в момент времени $t$ можно представить в виде суперпозиции простых гармонических волн:
\begin{equation}
\label{wave_height}
\begin{split}
h(\bvec{p}, t) = \sum_{i,j} a_{i,j}cos(\bvec{p}\cdot \bvec{k}_{i,j} - \omega(\bvec{k}_{i,j})t + \delta_{i,j}) \\
= Re\left( \sum_{\bvec{k}} \tilde{h}(\bvec{k}, t)
 \exp(i\bvec{k} \cdot \bvec{p}) \right)
\end{split}
\end{equation}
значения $\delta_{i,j}$, задающие фазу каждой гармоники, выбираются случайно.

Для быстрого сложения $N^2$ гармоник было применено быстрое обратное двухмерное преобразование Фурье. С его помощью эффективно вычисляются значения высот морской поверхности в узлах квадратной регулярной решетки размером $N$x$N$ и пространственной протяженностью $\dfrac{2\pi}{\Delta k}$. В используемой симуляции $N = 512$.

На рис. \ref{waveplot} представлены планшеты ядра БПФ и карт высот морского волнения для различных значение параметра формы углового распределения.

\begin{figure}[ht]
\begin{center}
\includegraphics[width=110mm]{wave_plot/spec_0}
\includegraphics[width=110mm]{wave_plot/spec_1}
\includegraphics[width=110mm]{wave_plot/spec_2}
\includegraphics[width=110mm]{wave_plot/spec_3}
\end{center}
\caption{Планшеты ядра БПФ и карт высот морского волнения для различных значение параметра формы углового распределения}
\label{boat_axis}
\end{figure}

