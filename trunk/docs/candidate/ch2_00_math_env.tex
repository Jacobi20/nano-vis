\section{Математическая модель нерегулярного волнения}

Используеvая модель волнения аппроксимирует морскую поверхность суперпизицией конечного числа гармонических волн \cite{lopatuhin2004}. Характер волнения определяется двухмерным энергетическим спектром $S(\omega, \theta)$, $\omega$ -- угловая частота волны, а $\theta$ -- угол, образуемый направлением бега волны с направлением ветра. В литературе \cite{lopatuhin2004} рассматриваются различные модели спектров волнения, выраженные в терминах функции энергитической плотности от угловой частоты и направления распространения. Примером такого спертра является спектр Пирсона-Московица:

\begin{equation}
\begin{split}
S(\omega, \theta) &= S(\omega) \; C \cos^m\theta \\
S(\omega) &= \dfrac{\alpha g^2}{\omega^5} \exp \left[ 
  -\dfrac{5}{4} \left( \dfrac{\omega_{peak}}{\omega} \right)^{4} 
\right]  \\
C &= \int_{-\pi/2}^{\pi/2} cos^m\theta d\theta
\end{split}
\end{equation}, 
где $m$ характеризует направленность спектра, $\omega_{peak}$ -- частоту пика спектра, а $\alpha = 0.0081$.

На практике непрерывный спектр волнения можно аппроксимировать конечным числом гармоник, спектральное распределение которых аппроксимирует используемый спектр. Для быстрого построения карты высот и скоростей для сложения гармоник используется двухмерное быстрое преобразование Фурье (БПФ) [].
 
При использовании БПФ оказалось удобнее представить энергитический спектр как функцию от волнового вектора гармоники $\vec{k}$:
$$ \vec{k} = k\vec{n},\quad 
k = \frac{\omega^2}{g} ,\quad 
\vec{n}=(cos \theta, sin \theta), \quad 
\omega(\vec{k}) = \sqrt{\lVert \vec{k} \rVert g} $$

Рассмотрим полную энергию волнения и произведем замену переменной интегрирования:

\begin{equation}
\begin{split}
E &= \int\limits_0^{2\pi}
     \int\limits_0^\infty S(\omega, \theta) d\omega d\theta
  = \int\limits_0^{2\pi}
     \int\limits_0^\infty 
         \frac {S(\sqrt{kg}, \theta)g} {2\sqrt{kg}} dk d\theta\\
  &= \int\limits_{\mathbb{R}^2}
         \frac{S(\omega(\vec{k}), \theta(\vec{k})) g}
              {2\omega(\vec{k}) \lVert \vec{k} \rVert} d\vec{k}
\end{split}
\end{equation}

%     &\approx \sum_{i=1}^N \hat{S}(i \Delta k)\Delta k
%     = \sum_{i=1}^N E_i     
 Тогда высоту морской поверхности в точке $(x, y)$ в момент времени $t$ можно записать ввиде



