\section{Графическая подсистема}

\subsection{Общая структура графической подсистемы}

Графическая подсистема логически разделена на два уровня абстракции:
\begin{itemize}
	\item	Драйвер (driver).
	\item	Сцена (scene).
\end{itemize}

Драйвер --- представляет собой слой абстракции от конкретного графического API.
Таким образом, это позволяет внедрять новые графические API не затрагивая сцену.
Основными сущностями драйвера, которыми он оперирует и предоставляет ну уровень выше --- сцене, являются:
\begin{itemize}
	\item	Эффект (Effect) --- представляет собой совокупность 
			состояний графического конвейера и набора из вершинного и пиксельного шейдера.
	\item	Вершинный буфер (Vertex buffer) --- представляет собой совокупность из набора массива вершин и массива индексов.
	\item	Текстура (Texture) --- двумерное (2D texture), трехмерное (Volume texture) или кубическое (Cube texture) изображение.
	\item	Внеэкранная поверхность (Render Target) --- может быть использована 
			как область, куда может осуществляться визуализация, а затем как двумерная текстура.
\end{itemize}


Как было отмечено выше графическая подсистема позволяет отображать следующие виды объектов:

\begin{itemize}
	\item	Твердые объекты (Solids).
	\item	Водная поверхность (Water).
	\item	Объемные скалярные поля (Volume Scalar fields).
	\item	Отладочные линии (Debug lines).
	\item	Элементы пользовательского интерфейса и текст.
	\item	Источники света.
\end{itemize}

Для отображения твердых объектов используется технология Deferred Shading \citep{stalker}, \citep{killzone}.
Для для закраски твердых объектов используется модель освещения Кука-Торренса, которая является физически обоснованной \citep{cook_torrance} и дает наиболее реалистичные блики \citep{ngan}.
Детальное описание техники визуализации твердых объектов, построения теней и расчета освещения представлено в статье \citep{bezgodov08}.
Пример построения теней представлен на рис.~\ref{shadows}.

\begin{figure}[ht]
\begin{center}
\includegraphics[width=150mm]{shadows}
\end{center}
\caption{Построение теней от элементов надстройки корабля}
\label{shadows}
\end{figure}



Для отображения элементов пользовательского интерфейса и текста используются списки наборы четырехугольников с наложенными текстурами.
Для отображения текста используются растровые шрифты, которые могут быть подготовлены из TrueType или OpenType шрифтов с использованием утилиты Bitmap Font Generator \citep{bmfont}.

Для отображения отладочной информации графическая система предоставляет функциональность по рендерингу трехмерных линий, 
а также следующих объектов, составляемых из линий:

\begin{itemize}
	\item	Стрелки (варьируются цвет, длина и форма наконечника)
	\item	Точки (изображаются как три взаимно-пересекающихся отрезка, варьируются размер и цвет)
	\item	Ориентированные по осям проволочный параллелепипед (варьируются размер и цвет)
\end{itemize}

Отладочные линии, а также их составные из линий объекты могут использованы для визуализации такой информации как:
\begin{itemize}
	\item	Силы и моменты действующие на судно.
	\item	Линии тока.
	\item	Траектории движения и графики.
\end{itemize}

На рис.~\ref{super_plot} представлен пример отладочной визуализации и текста.

\begin{figure}[ht]
\begin{center}
\includegraphics[width=110mm]{super_plot}
\end{center}
\caption{Отладочная визуализация и текст}
\label{super_plot}
\end{figure}

Для отображения трехмерных скалярных полей используется технология объемного рендеринга, которая 
заключается в трассировке луча в пространстве объемной текстуры и последующим накоплением 
значения трансфер-функции вычисленной для данной точки объема \citep{Engel01high-qualitypre-integrated}.
Пример объемного рендеринга на примере магнитных полей представлен на рис.~\ref{volume}

\begin{figure}[ht]
\begin{center}
\includegraphics[width=110mm]{volume}
\end{center}
\caption{Пример рендеринга скалярных полей}
\label{volume}
\end{figure}

\subsection{Морфологическое сглаживание степенчатости изображения}
В широкоэкранных системах ВР остро встает проблема «ступенчатости» изображений, которая обусловлена конечным размером пикселя. Например, для экрана системы ВР ЦСМВ СПбГУ ИТМО размер пикселя при разрешении 1920х1080 будет составлять около 2 мм, что визуально ощутимо на небольшом расстоянии от экрана. 

Для того чтобы избавиться от этого эффекта и сделать изображение более реалистичным используются специальные техники сглаживания.
На данный момент существует два наиболее часто используемых метода сглаживания, это избыточная выборка сглаживания (англ. Super Sampling anti-aliasing, SSAA) и множественная выборка сглаживания (англ. Multisample anti-aliasing, MSAA). Метод SSAA заключается в том, что вначале синтезируется изображение, в несколько раз превосходящее по размерам финального изображение, после чего это изображение сжимается до размеров финального, при этом происходит усреднение всех соседних пикселей. В результате работы SSAA получается наиболее качественное изображение, но данный метод крайне требователен к производительности системы и к её памяти, а в случае визуализации стереоизображения эти требования возрастают в два раза, что является мало приемлемым. Метод MSAA аналогичен SSAA, с той лишь разницей, что расчет цвета пикселя осуществляется один раз и записывается сразу в несколько суб-пикселей.

Следует отметить, что необходимость использования сглаживания и синтеза изображения высокого разрешения, а также синтез парных изображений (для создания эффекта стерео) ставит ограничение на возможность использования SSAA и MSAA, так как сильно возрастает объем памяти необходимый для буфера изображения. В связи с такими ограничениями было решено использовать метод морфологического сглаживания (англ. Morphological Antialiasing, MLAA)\citep{mlaa}. МLAA работает со сценами любой сложности и с любой техникой, фактически данный метод работает только с финальным изображением, он не настолько требователен к объему памяти как вышеперечисленные методы, а результат в большинстве случае не уступает результату работы метода SSAA. 

Данный метод заключается в нахождении \glqq L\grqq-образных форм на \frqt{ступенчатых} разрывах непрерывности и их размытии. Метод MLAA можно условно разделить на три этапа: 
\begin{enumerate}
\item	На этом этапе находятся все разрывы непрерывности в изображении, 
		точность на этом этапе можно повысить благодаря использованию Z-буфера. 
		На этом этапе можно применить любой метод нахождения разрывов непрерывностей 
		(граней, ребер) из области компьютерного зрения.
\item	На втором этапе рассчитывается длина найденных непрерывностей, это необходимо 
		для расчета уровня сглаживания пикселей. 
\item	На третьем этапе происходит поиск всех \glqq L\grqq-образных форм и их сглаживание.
\end{enumerate}

На рис.~\ref{mlaa} приведен пример синтезированного изображения без сглаживания и с использованием с MLAA.

\begin{figure}[ht]
\begin{center}
\includegraphics[width=140mm]{mlaa}
\end{center}
\caption{Результат применения морфологического сглаживания}
\label{mlaa}
\end{figure}





%---------------------------------------------------------------------------------
%	WATER RENDERING :
%---------------------------------------------------------------------------------

\subsection{Технология отображения морского волнения}

Для создания эффекта присутствия необходимо отображать визуально бесконечную водную поверхность простирающуюся о точки наблюдения до горизонта. Для визуализации безграничного моря используются две основные техники:
\begin{enumerate}
\item	Сетки в пространстве экрана (Screen space grids) \citep{projgrid}.
\item	Неравномерные, привязанные к камере сетки \cite{crysis}.
\end{enumerate}

В ходе работы были реализованы оба варианта визуализации водной поверхности. Вариант с использованием неравномерных сеток привязанных к камере оказался более стабильным и с небольшими изменениями используется как основной.
Для отображения используется сетка подготовленная особым образом. cм. рис.~\ref{sea_surface_twin}. Сетка имеет следующую структуру: 

\begin{itemize}
\item	А - \frqt{дно} - используется для маркировки буфера трафарета при отображении раздела сред; 
\item	B - \frqt{область волнения} - используется для отображения волн; 
\item	C - \frqt{область горизонта} - область, которая находится достаточно далеко от наблюдателя и видимой высотой волн можно пренебречь.
\end{itemize}

\begin{figure}[ht]
\begin{center}
\includegraphics[width=140mm]{sea_surface_twin}
\end{center}
\caption{Сетка для отображения морской поверхности. Слева – общая структура сетки, справа – сетка области интенсивного волнения. (обозначения – по тексту)}
\label{sea_surface_twin}
\end{figure}

Построение элементов сетки \frqt{A} и \frqt{С} --- тривиально. Построение области \frqt{A} осуществляется следующим образом:
\begin{enumerate}
	\item	Строится сетка размером состоящая из 16 квадратов.
	\item	Четыре центральных квадрата разбиваются еще на 4 квадрата каждый.
	\item	Шаг №2 повторяется 5-6 раз.
	\item 	К полученной сетке применяется алгоритм сглаживания полигональных сеток Катмулла-Кларка \citep{catmull_clark}.
\end{enumerate}

Полученная таким образом сетка обладает преимуществом перед сетками составленным из квадратных патчей, которое заключается в сглаживании границы перехода между более детальным и менее детальным участком сетки.

При визуализации моря, центр сетки всегда находится под или над камерой. Поворот камеры на ориентацию сетки в пространстве не влияет. Высота вершин в области «B» модифицируется вершинным шейдером на GPU в соответствии с моделью волнения. При этом, высота волн плавно уменьшается по мере увеличения расстояния от наблюдателя. Карта высот волн сгенерированная с использованием FFT на CUDA записывается в текстуру (этот процесс осуществляется полностью на GPU) и полученная текстура используется вершинным шейдером для модификации высоты вершин. Для эффекта отражения Френеля необходимо знать значение нормали в каждой точке. Для этого карта высот волн сэмплируется в нескольких точках и нормаль получается методом конечных разностей.


% ФРЕНЕЛЬ %

При закрашивании водной поверхности учитывается частичное отражение Френеля (только небо, которое задается панорамной текстурой), 
и частичное преломление с экспоненциальным затуханием по глубине. См. рис.~\ref{water_effects}.

\begin{figure}[ht]
\begin{center}
\includegraphics[width=140mm]{water_effects}
\end{center}
\caption{Визуализация поверхнсти воды: отражение Френеля (слева), экспоненциальное затухание с глубиной (справа).}
\label{water_effects}
\end{figure}


% РАЗДЕЛ СРЕД %

Следует отметить, что камера может находиться не только над водой, но и под водой, а также, на границе сред. Для корректного отображения границы сред используется следующая техника:
\begin{enumerate}
\item	При визуализации морской поверхности включается запись в буфер трафарета: каждый раз растеризуется треугольник значение соответствующего бита в буфере трафарета инвертируется. Таким образом, если пиксель находится внутри сетки, то значение в буфере трафарета будет равно 1, и 0 – если пиксель находится за пределами сетки.
\item	Для всех пикселей, для которых значение в буфере трафарета равно 1 применяется эффект затуманивания.
\end{enumerate}

Изображение полученное при позиционировании камеры на граница сред представлено на рис.~\ref{water_fog}.

\begin{figure}[ht]
\begin{center}
\includegraphics[width=140mm]{water_fog}
\end{center}
\caption{Визуализация границы сред}
\label{water_fog}
\end{figure}


% КОРАБЕЛЬНЫЕ ВОЛНЫ %

Дополнительным аспектом отображения визуальных свойств водной поверхности является воспроизведение расходящихся корабельных волн при движении плавучего морского объекта. Поскольку нелинейная задача расчета корабельных волн в полной постановке является существенно более ресурсоемкой по сравнению с моделью (??????), для создания визуального эффекта используется упрощенная модель, основанная на решении линейного уравнения колебаний на регулярной сетке:

\begin{equation}
	\frac{d^2U}{dt^2} = \frac{d^2U}{dx^2} +  \frac{d^2U}{dy^2}
\end{equation}

Начальное возмущение формируется в точке пересечения корпуса корабля и водной поверхности. Величина возмущения определяется в зависимости от относительной скорости поверхности корабля относительно воды. Уравнение решается на GPU с использованием библиотеки CUDA, результат расчета как набор значений в текстуре передается в вершинный шейдер и высота колебаний добавляется к высоте волн. В тех областях, где скорость частиц колеблющейся поверхности выше определенного значения, поверхность моря перекрашивается в белый цвет, что дает эффект пены. См. рис.~\ref{wakes}.

\begin{figure}[ht]
\begin{center}
\includegraphics[width=140mm]{wakes}
\end{center}
\caption{Корабельные волны)}
\label{wakes}
\end{figure}

Корабельные волны в данной реализации являются только визуальным эффектом и на ход моделирования поведения судна не влияют.




