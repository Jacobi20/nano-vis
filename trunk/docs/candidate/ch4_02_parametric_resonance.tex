\section{Режим параметрического резонанса лагом к волне}

Параметрический резонанс является одним из наиболее интересных явлений экстремальной динамики судна.
Интерес обусловлен как редкостью явления, так и его опасностью. 
Одним из ярчайших примеров аварий вызванных параметрическим резонансом является 
авария контейнеровоза класса C11 водоизмещением более 150 000 тонн, при которой 
угол крена достигал $40^{\circ}$, и, как следствие, было потеряно более $2/3$ 
груза \citep{c11}.

Параметрический резонанс обуславливается периодическим изменением характеристик остойчивости судна (\frqt{валкость}), что, при совпадении частот изменения остойчивости судна  и частоты собственных 
бортовых колебаний может привести к возникновению резонансу. 


%Основной резонанс качки судна является, одним из основных классов экстремальных явлений, связанных с интенсивными колебаниями в одной из плоскостей. Сама по себе сильная качка не является угрожающей для судна; скорее, она сказывается на обитаемости и на эксплуатационных характеристиках. Однако нахождение судна в режиме основного резонанса может стать причиной возникновения цепочки неблагоприятных последствий. Например, при сильной бортовой качке малого судна возможно резкое ухудшение остойчивости при попадании воды в палубный колодец, при килевой качке – сильный слеминг и пр. Избежать резонансных условий позволяет изменение скорости и (или) курсового угла судна, осуществляемое обычно посредством операционной диаграммы. В данном разделе воспроизведение резонансных режимов качки рассматривается как базовый тест на работоспособность виртуального полигона.

\subsection{Цель и постановка эксперимента}

\textbf{Условием} возникновение параметрического резонанса  лагом к волне является периодическое изменение характеристик остойчивости судна обусловленное изменением осадки, (которое, в свою очередь, может быть обусловлено вертикальной качкой). Если выполняется условие $2 \omega_{roll} \approx \omega_{heave} \approx \omega_{max}$, то судно может попасть в режим параметрического резонанса.

\textbf{Целью} эксперимента является определение и сравнение опасных диапазонов частот морского волнения.

Для проведения эксперимента используется модель судна класса \frqt{катер} со следующими характеристиками:
\begin{itemize}
	\item	$L = 40\ \text{м}$
	\item	$B = 12\ \text{м}$
	\item	$T = 3\ \text{м}$
	\item	$D = 600\ 000\ \text{кг}$
\end{itemize}

Эксперимент состоит из следующих этапов:
\begin{enumerate}
	\item	Определение частот собственных колебаний судна 
			$\omega_{roll}$, $\omega_{pitch}$, $\omega_{heave}$ путем кренования на тихой воде.
	\item	Выбирается набор частот пика спектра волнения $(m, \omega_{max})$, для которых
			проводятся запись качки судна, в течении 900 секунд.
	\item	Этап №2	проводится для всех значений $\omega_{max}$
			
			$\omega \in [0.8..3.0],\ \Delta\omega=0.05$ 

\end{enumerate}

\subsection{Анализ результатов эксперимента}

Согласно результатам эксперимента были получены следующие частоты собственных колебаний (временные диаграммы качки показаны на рис.~\ref{exp_pr_rolling_lag}):
\begin{itemize}
	\item	$ \omega_{roll} = 1.14 \text{с}^{-1}$
	\item	$ \omega_{pitch} = 3.25 \text{с}^{-1}$
	\item	$ \omega_{heave} = 2.36 \text{с}^{-1}$
\end{itemize}

\begin{figure}[ht]
	\begin{center}
	\includegraphics[width=120mm]{exp_pr_lag/rolling}
	\includegraphics[width=120mm]{exp_pr_lag/pitching}
	\includegraphics[width=120mm]{exp_pr_lag/heaving}
	\end{center}
	\caption{Временные диаграммы бортовой, килевой и вертикальной качки.}
	\label{exp_pr_rolling_lag}
\end{figure}

Для всех параметров морского волнения определяется 10\%-ная обеспеченность (90\%-ная квантиль) модуля угла бортовой качки и делится на корень общей энергии волнения. График данного соотношения (исходные результаты) в зависимости от частоты пика спектра волнения показаны на рис.~\ref{exp_pr_rolling_resonance_lag}.

На графике присутствуют два вида резонанса:
\begin{itemize}
	\item	Основной --- приходится на частоты $1.0..1.5$.
	\item	Параметрический --- приходится на частоты $2.1..2.2$.
\end{itemize}


На основании обработки экспериментальных данных можно сделать следующие выводы:
\begin{enumerate}
	\item	При параметрическом резонансе потери энергии весьма малы по сравнению с основным резонансом. 
			Также судно, как колебательная система, при параметрическом резонансе обладает большей частотной избирательностью.
\end{enumerate}

Снимок экрана ВП в процессе моделирования параметрического резонанса представлен на рис.~\ref{exp:pr_movie}.

\begin{sidewaysfigure}
	\begin{center}
	\includegraphics[width=230mm]{exp_pr_lag/roll_resonance}
	\end{center}
	\caption{Основной и параметрический резонанс бортовой качки}
	\label{exp_pr_rolling_resonance_lag}
\end{sidewaysfigure}


\begin{sidewaysfigure}
	\begin{center}
	\includegraphics[width=200mm]{exp_pr_lag/movie}
	\end{center}
	\caption{Опасный крен вызванный параметрическим резонансом}
	\label{exp:pr_movie}
\end{sidewaysfigure}

